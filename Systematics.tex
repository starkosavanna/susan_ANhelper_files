\section{Systematics}\label{sec:systematics}
%\textbf{Responsible: T. Kamon, E. Laird, M. Dalchenko}
%\iffalse
The following systematics have been considered (summarized in Table~\ref{table:SystematicsTable}):

\begin{itemize}
%  \item \textbf{Parton Distribution Functions (PDF):} The systematic effect due to imprecise knowledge of the parton 
%distribution functions is determined by following the PDF4LHC recommendation and comparing various PDF sets with the 
%CTEQ6.6L, MSTW2008nnlo, and NNPDF20 PDF with the 
%default PDF and variations within the family of parametrization \cite{CTEQ}. The maximal deviation from the central value is used the overall 
%systematic due to PDFs. We obtain a value of 6.5\%.
%
%  \item \textbf{Initial State Radiation (ISR) and Final State Radiation (FSR):} The systematic effect due to imprecise 
%modeling of initial and final state radiation is determined by re-weighting events to account for effects such as 
%missing a terms in the soft-collinear approach \cite{softCollinear} and missing NLO terms in the parton shower approach \cite{partonShower}. We 
%obtain uncertainties of 0.9\% and 1.2\% for ISR and FSR respectively.

  \item \textbf{Luminosity:} We consider a 2.5\% uncertainty on the measured luminosity \cite{REFLUMI}. It is considered 100\% correlated across MC based 
backgrounds within a channel. It is also considered 100\% correlated across channels (for MC based backgrounds).

  \item \textbf{Trigger, Reconstruction, and Selection:} 
  An overall uncertainty is applied for the trigger uncertainties determined on the 
  correction factors described in Section 3 which are measured using tag-and-probe methods. 
  %We consider 6.8\% uncertainty per hadronic tau leg \cite{CMS-PAS-TAU-11-001}.
  %Scale factors for $\tau_{h}$ identification are taken from the tau POG and obtained using a fit of data in a Z$\to\tau\tau$ enhanced region and fixing the cross section to that measured using ee/$\mu\mu$. 
  %We consider 5\% uncertainty per hadronic tau leg in 
  To be conservative, we assign a 5\% systematic uncertainty, per $\tau_{h}$ leg, on the trigger efficiency in the $\tau_{h}\tau_{h}$ channel due to the ineffiency of the trigger at $p_{T} > 70$ GeV. 
  %the $\tau_{h}\tau_{h}$ channel 
  Each leg is 100\% correlated to the other. The trigger uncertainty is considered 100\% correlated across MC based backgrounds within 
a channel. It is also considered 100\% correlated across channels using the same trigger. For the case of the $\tau_{h}\tau_{h}$ trigger, where the trigger 
efficiency uncertainty is measured per $\tau_{h}$ leg, the total trigger uncertainty is calculated by assuming both legs are 100\% correlated. For example, if the 
per leg $\tau_{h}$ trigger uncertainty is 5\%, the total trigger uncertainty for the $\tau_{h}\tau_{h}$ channel will be 10\% ($2\cdot 5$\%). A 5\% uncertainty is also applied 
in the light lepton channels. As described in Section 3, the trigger turn-on curves in data and MC are divided, resulting in bins of data/MC vs $p_{T}$. These points are 
fit to a flat line to extract a global SF to correct the emulation of the trigger efficiency in MC. This fit has an uncertainty (stats) of $\sim 3-4$\%. However, to be conservative we 
assign a 5\% systematic uncertainty per electron/muon leg, where 5\% is the maximal deviation between a data/MC point and the flat fit line. 

  \item \textbf{$b$-Tagging Efficiency:} We consider a 30\% uncertainty on the mis-tag rate as measured by the 
b-tagging POG \cite{CMS_PAS_BTV_11-001}. For the case of our signal, the systematic 
uncertainty on the requirement of 0 jets mis-tagged as b-jets is determined by propagating the 30\% uncertainty on the 
mis-tag rate through the following equation (which represents the signal efficiency for requiring 0 
jets mis-tagged as b-Jets):

\begin{equation}\label{eq:nttbar}
  \epsilon^{\textrm{NBtag} < 1} = 1 - \sum_{n=1} P(n) \cdot \sum_{m=1}^{n} C(n,m) \cdot f^{m} \cdot (1-f)^{n-m}
\end{equation}

where $P(n)$ is the probability to obtain $n$ additional jets (non-tau and non-lepton) in the event, $C(n,m)$ the 
combinatorial of $n$ $choose$ $m$, and $f$ the mis-tag rate. The probability to 
obtain at least one additional jet in the event is $\sim$ 10\%. Therefore, based on the above equation, the 
mis-tag rate and uncertainty, and the probability to obtain at least one additional jet we calculate a 
systematic effect of $\sim 5$\% on our signal due to the mis-tag rate. The b-tagging/mis-tagging systematics are considered 100\% correlated across MC based 
backgrounds with similar composition (e.g. W + jets and DY + jets where there are typically no real b-jets), but completely uncorrelated to backgrounds that have 
different composition (e.g. $t\bar{t}$ vs. DY + jets).

  \item \textbf{Electron Energy Scale:} We consider the effect on the signal acceptance efficiency of a 1\% (2.5\%) shift on the electron
  energy scale in the barrel (endcap) region. The resultant systematic uncertainty on signal and MC based backgrounds is $< 1$\%.

  \item \textbf{Electron Identification + Trigger:} We consider a 6\% uncertainty on the combination of electron reconstruction, identification and the single 
electron trigger as measured with a data-driven method \cite{electron8TeV, ZpCMSmumuee13}.

  \item \textbf{Muon Momentum Scale:} We consider the effect on the signal acceptance efficiency of a 1\% momentum scale uncertainty on the
  muon momentum. The resultant systematic uncertainty on signal and MC based backgrounds is $< 1$\%.

  \item \textbf{Muon Identification + Trigger:} We consider a 7\% uncertainty on the combination of muon reconstruction, identification and the single muon 
trigger as measured with a data-driven method \cite{muon13TeV, ZpCMSmumuee13}.

  \item \textbf{Tau Energy Scale:} We consider the effect of the 3\% tau energy scale uncertainty measured by the tau 
POG on the signal acceptance. The tau 4-momentum is scaled by a factor of $k=1.03$ ($p_{smeared} = k \cdot 
p_{default}$) and variables are recalculated using $p_{smeared}$. We find that by using $p_{smeared}$ calculated with 
a factor of $k=\pm 1.03$, the signal and MC baded backgrounds fluctuates by up to $\sim 11$\%. 

 \item \textbf{Tau Identification:} We consider a 5\% uncertainty per $\tau_{h}$ on the $\tau_{h}$ identification.
A dominant systematic uncertainty is related to the confidence that the MC simulation correctly models the identification 
efficiency for high-\pt $\tau_{h}$ candidates. This additional uncertainty per $\tau_{h}$ amounts to 35\% $\cdot$ \pt / 1 \TeV, 
resulting in $<$ 4\% (10\%) uncertainty for a reconstructed mass of 500 GeV in the $\tau_{\ell} \tau_{h}$ ($\tau_{h}\tau_{h}$) 
channel and $<$ 12\% (25\%) at 2 TeV. Technically the systematic uncertainty on high-$p_{T}$ $\tau_{h}$ identification is determined by 
applying the $p_{T}$ dependent weights, per leg, and extracting varied mass templates which are used as input in the fitter as 
shape systematics. 

  \item \textbf{Jet Energy Scale:} We consider the effect of a 3-5\% jet energy scale uncertainty on the signal 
acceptance (depending on the $\eta$ and $p_{T}$ of the considered jet as prescribed by the $JetMET$ $POG$). For example, for a 5\% JES uncertainty, the jet 
4-momentum is scaled by a factor of $k=1.05$ or $0.95$ ($p_{smeared} = k \cdot p_{default}$) and variables are recalculated 
using $p_{smeared}$. We find that by using $p_{smeared}$ calculated with
a factor of $k=\pm 1.05$ or $0.95$, the signal and MC baded backgrounds fluctuates by up to $\sim 12$\%. 
  \item \textbf{MET:} The uncertainty on MET for our signal process is driven by the tau energy scale (TES), jet energy 
scale (non-tau jets) (JES), light lepton energy/momentum scale (LES), and unclustered energy (UCE). We follow the official recommendation. 
The systematic effect from MET due to TES, JES and LES is included in the JES, 
TES, LES systematic uncertainties described above. We find that a 10\% uncertainty on the unclustered energy results 
in at most a 0.5\% fluctuation on the signal acceptance and MC-based BG predictions.
%
%  \item \textbf{MET:} The uncertainty on MET for our signal process is driven by the jet energy 
%scale (non-tau jets) (JES), light lepton energy/momentum scale (LES), and unclustered energy (UCE).
%The systematic effect from MET due to TES, JES and LES is included in the JES,
%TES, LES systematic uncertainties described above. We find that a 10\% uncertainty on the unclustered energy results
%in at most a 0.5\% fluctuation on the signal acceptance.

  \item \textbf{PDF Systematics Uncertainty:} We consider the effect of the PDF uncertainties on the signal acceptance 
by using the mass binned DY samples to mock up Z' samples in the following way:
\begin{table}[ht]	
  \begin{tabular}{| l | c | c | c | c | c | c |} 
  \hline\hline 
  Z' mass point (GeV) & 500 & 1000 & 1500 & 2000 & 2500 & 3000\\
  \hline
  DY mass slice & 500to550 & 1000to1050 & 1500to1550 & 2000to2050 & 2450to2550 & 2800to3000\\
  \hline \hline
  \end{tabular}
\end{table}

Following the "PDF4LHC recommendations for LHC Run
II"~\cite{PDF4LHC15}, the PDF uncertainties are computed from the
68$\%$ confidence level with the PDF4LHC15$\_$mc sets. The PDF 
uncertainties for our main backgrounds, \ttbar, W+Jets and DY, are
much smaller than their bin-by-bin statistical uncertainties thus is
neglected. The PDF uncertainties on the signal acceptance range from
0.7$\%$ for Z' at 500 GeV up to 12$\%$ for Z' at 3 TeV.

  \item \textbf{Pileup:} We have followed the recommendation for pileup corrections and used a 69.2 mb minbias cross-section. We have considered two approaches to 
determine systematic uncertainty due to pileup. The first method, which is the official recommendation, is to vary the minbias cross-section by 5\% and re-derive pileup 
weights. Using the new weights with $\pm$5\% in the minbias cross-section results in at most a 2\% change in the signal and MC-based BG yields over the full mass range. 
As a more conservative approach, we take the difference in yields and shapes between ``with pileup weights'' and ``without pileup weights''. This amounts to at most 5\% 
uncertainty in signal or backgrounds. 
%However, we find that 71 mb provides a
%better modeling of the number of privary vertices observed in data (Figure~\ref{fig:69vs71mb}). While this distinction is not particularly important for the
%backgrounds, since they are determined from data, we assign a systematic uncertainty on signal due to pileup. The uncertainty is determined as the difference in
%signal yields between the 69 mb and 71 mb scenarios. For all signal masses, the uncertainty is 1\%.

  \item \textbf{Background Estimates:} The uncertainty on the data-driven background estimations are driven by the statistics in data in the various control
samples. 
%There is also a mostly negligible contribution from the level of contamination
%from other BGs in the control regions. 
In cases where MC based BGs must be subtracted off, the uncertanties in the MC bgs due to the above listed
systematic uncertanties are propagated throughout the subtraction and used to assign a systematic uncertainty on the background prediction. 
Table~\ref{table:SystematicsTable} summarizes the systematic uncertainties considered. The ``Closure+Norm'' row represents the uncertainty in 
the BG prediction due to the extracted data-to-MC SFs in control regions or the transfer factors used to derive full data-driven estimates. 
In summary, the systematic uncertainties range from 2-20\%, depending on the background. For example, the DY$\to\tau\tau$ control regions 
result in data-to-MC scale factors consistent with 1, but with uncertainties between 3\% and 5\%, depending on the channel. The $t\bar{t}$ scale 
factors range from 1\% in the $e\mu$ channel to 6\% in the $e\tau_{h}$ channel. A conservative estimate of the uncertainty on the diboson background 
is derived from a $e\mu$ control sample which results in a scale factor of $1.01 \pm 0.20$, and thus a 20\% uncertainty on diboson is assigned in all 
channels. The systematic uncertainty on the QCD prediction in the $\tau_{h}\tau_{h}$ channel is dominated by the measured OS-to-SS transfer factor 
of $1.32 \pm 0.12$ (9\% relative uncertainty). In the $\ell\tau_{h}$ channels, the uncertainty on the QCD prediction is dominated by the uncertainty 
on the ``Tight-to-Loose'' ratio $f_{LT}^{QCD}$, with an uncertainty of 
11\% in the $\mu\tau_{h}$ channel, 8\% in the $e\tau_{h}$ channel, and 
23\% in the $e\mu$ channel. The uncertainty on the W + jets estimate is dominated by the uncertainty on the measurement of the ``jet-to-$\tau_{h}$'' fake 
rate used to extrapolate to the signal region. This results in an 
uncertainty of 8\% in the $\mu\tau_{h}$ channel, 15\% in the $e\tau_{h}$ 
channel, and 33\% in the $e\mu$ channel. 

\end{itemize}

\newcommand{\ch}{\tiny $hh$, $\mu h$, $eh$, $e\mu$}
\newcommand{\chs}{\footnotesize \thth, \tmth, \teth, \tetm}

\begin{table}[ht]\centering
 \caption{Summary of systematic uncertainties. Values are given in
   percent.  ``s'' indicates template variations (``shape''
   uncertainties). $\rm L=2.5\%$.}
 \resizebox{\textwidth}{!}{
 \begin{tabular}{|l|c|c|c|c|c|c|} \hline \hline
   Source                  & QCD         & W           & DY          & \ttbar      & VV          & Signal      \\
   \chs:                   & \ch         & \ch         & \ch         & \ch         & \ch         & \ch         \\
   \hline Lumi             & --,--,--,-- & L,--,--,L   & L,L,L,L     & L,L,L,L     & L,L,L,L     & L,L,L,L     \\
   \hline $\mu$ ID + Trig  & --,--,--,-- & --,--,--,7  & --,7,--,7   & --,7,--,7   & --,7,--,7   & --,7,--,7   \\
   \hline e ID + Trig      & --,--,--,-- & --,--,--,6  & --,--,6,6   & --,--,6,6   & --,--,6,6   & --,--,6,6   \\
   \hline $\tau_{h}$ Trig  & --,--,--,-- & 10,--,--,-- & 10,--,--,-- & 10,--,--,-- & 10,--,--,-- & 10,--,--,-- \\
   \hline $\tau_{h}$ ID    & --,--,--,-- & 10,--,--,-- & 10,6,6,--   & 10,6,6,--   & 10,6,6,--   & 10,6,6,--   \\
\hline high-$p_{T}$ $\tau_{h}$ ID    & --,--,--,-- & $<5$,--,--,-- & $<10$,$<5$,$<5$,--   & $<10$,$<5$,$<5$,--   & $<10$,$<5$,$<5$,--   & $<30$,$<15$,$<15$,--   \\
   \hline b ID             & --,--,--,s  & 10,--,--,s  & 5,5,s,s     & 10,12,s,s   & 5,5,s,s     & 5,5,s,s     \\
   %% \hline JES           & --,--,--,s  & 12,--,--,s  & 8,9,s,s     & 12,10,s,s   & 8,9,s,s     & 2,2,s,s     \\
   \hline JES              & --,--,--,s  & 12,--,--,s  & 8,s,s,s     & 12,s,s,s    & 8,s,s,s     & 2,2,s,s     \\
   %% \hline TES           & --,--,--,s  & 11,--,--,s  & 11,7,s,s    & 11,9,s,s    & 8,8,s,s     & 3,3,s,s     \\
   \hline TES              & --,--,--,s  & 11,--,--,s  & 11,s,s,s    & 11,s,s,s    & 8,s,s,s     & 3,3,s,s     \\
   \hline MMS              & --,--,--,-- & --,--,--,1  & --,1,--,1   & --,1,--,1   & --,1,--,1   & --,1,--,1   \\
   \hline EES              & --,--,--,-- & --,--,--,1  & --,--,1,1   & --,--,1,1   & --,--,1,1   & --,--,1,1   \\
   \hline top~\pt          & --,--,--,-- & --,--,--,-- & --,--,--,-- & --,--,--,s  & --,--,--,-- & --,--,--,-- \\
   \hline pdf              & --,--,--,-- & --,--,--,-- & --,--,--,-- & --,--,--,-- & --,--,--,-- & (1--12)     \\
   \hline bin-by-bin stat. & s,s,s,s     & s,s,s,s     & s,s,s,s     & s,s,s,s     & s,s,s,s     & s,s,s,s     \\
   \hline Closure+Norm.    & 9,11,8,23 & --,8,15,33    & 8,3,5,5  & --,4,6,1     & 20,20,20,20 &             \\
%   \hline W+Jets MC Norm.  & --,--,--,-- & --,--,7,--  & --,--,--,-- & --,--,--,-- & --,--,--,-- & --,--,--,-- \\
   \hline \hline
 \end{tabular}
 \label{table:SystematicsTable}
}
\end{table}


