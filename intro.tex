\section{Introduction}\label{sec:intro}
% \textbf{Responsible: A. Gurrola, T. Kamon, F. Romeo, E. Laird} 
%\iffalse

At the forefront of the interconnection between particle physics and cosmology are the following questions: (1) What is the origin of the matter-antimatter 
asymmetry?; (2) What is the origin of neutrino mass?; (3) Are there new fundamental forces in nature?; (4) What is the origin of dark energy; and (5) Is the Higgs 
boson solely responsible for electroweak symmetry breaking and the origin of mass? Much like the Higgs mechanism is introduced to account for the SU(2)xU(1) 
symmetry breaking, there are a plethora of theoretical models which incorporate additional gauge fields and interactions to address these questions. For example, 
string theory is considered a promising candidate for describing gravitational systems at strong coupling and thus plays a prominent role in the description of 
black holes and evolution of the universe through the understanding of the origin of dark energy. Similarly, models with additional neutrino fields at the TeV 
scale provide a possible explanation for the mass of light neutrinos. Such models often manifest themselves as new heavy particles that could be observed at the 
LHC. Surprisingly, some of these new particles predicted on the basis of pure particle physics arguments can even provide the correct dark matter relic density. 
There are several ways new heavy gauge bosons appear. The most natural possibility is one in which these heavy gauge bosons are the gauge field of a new 
local broken symmetry. Examples include models with a new U(1) gauge symmetry, little Higgs models, and E6 Grand Unified Theories (GUT). 
In models with a new U(1) gauge symmetry, the $Z^\prime$ is the gauge boson of the broken symmetry. In
little Higgs models, breaking of the global symmetry by gauge and Yukawa interactions generates Higgs mass and couplings at the TeV scale that cancel off the SM
quadratic divergence of the Higgs mass from top, gauge, and Higgs loops. This results in one or more $Z^\prime$ bosons. In Kaluza-Klein models, the $Z^\prime$
bosons are excited states of a neutral, bulk gauge symmetry. From the breadth, scope, and implications of these models, it is apparent that probing these questions and puzzles potentially lies in 
the physics of new particles at the TeV scale. 

Of particular interest for this analysis note are models that include an extra neutral gauge boson that decays to pairs of high-$p_{T}$ $\tau$ leptons. 
In particular, extensions to the SM proposed as an explanation for the high mass of the top quark predict $Z^\prime$ bosons that typically couple to
third-generation fermions. 
%~\cite{Lynch}.
Examples are the topcolor-assisted technicolor (TAT) models, where the $Z^\prime$ is denoted as $Z^{\prime}_{\textrm{TAT}}$ 
%~\cite{TopcolorAssistedTechnicolor1,TopcolorAssistedTechnicolor2}.
Additionally, models with enhanced $Z^\prime$ couplings to third-generation fermions exist to explain the B-meson anomalies observed at LHCb. 
%cite{}
Although many models with extra gauge bosons obey the universality of the couplings, models with generational dependent couplings resulting in extra neutral gauge 
bosons that preferentially decay to $\tau$ leptons make this analysis an important mode for discovery. 

However, even if nature has chosen a model with universal couplings and thus a new gauge boson decaying to $\mu\mu$ or ee is discovered first, it will be critical to establish the $\tau\tau$ decay 
channel to establish the coupling relative to the $\mu\mu$ and ee channels. For this reason, a widely used benchmark model in searches for $Z^\prime$ bosons is the sequential
standard model (SSM), 
%~\cite{Altarelli}
which predicts a neutral spin-1 $Z^\prime$ boson, 
denoted $Z^{\prime}_{\textrm{SSM}}$, with the same couplings to quarks and leptons as the SM Z boson. Results for models with generational dependent couplings may (in many cases) be determined by 
appropriately 
scaling the the SSM cross-section times branching ratio. 

Results of searches for heavy $\tau\tau$ resonances in proton-proton (pp) collisions at either $\sqrt{s}=7$, 8, or 13\TeV have been reported by the ATLAS and CMS Collaborations and exclude 
$Z^{\prime}_{\textrm{SSM}}$ ($Z^{\prime}_{\textrm{TAT}}$) masses below 2.1 (1.7)\TeV. 
%~\cite{ZpCms7,ZpAtlas7, ZpAtlas8}
The most
stringent mass limits on $Z^{\prime}_{\textrm{SSM}}$ production, set by ATLAS and CMS
in searches for a narrow resonance decaying into an 
ee or $\mu\mu$ pair, are 3.4 
%~\cite{Aaboud:2016cth}
and 3.2
%~\cite{ZpCMSmumuee13}
\TeV, respectively.

In this iteration of the analysis we report on a search for physics beyond the SM in
events containing a pair of high transverse momentum (p$_{T}$) oppositely
charged $\tau$ leptons. 
%Four $\tau\tau$ final states, $\tau_\Pe \tau_{\mu}$, $\tau_\Pe \tauh$, 
%$\tau_{\mu} \tauh$, and \ditauhad, are selected, where $\tau_\ell$ ($\ell = \Pe,\mu$) and \tauh
%refer to the leptonic and hadronic decay modes of the $\tau$ lepton, respectively.
The study is based on a data sample of
pp collisions at $\sqrt{s} = 13\TeV$ recorded in 2016 with the CMS detector at the LHC.
The sample corresponds to an integrated luminosity of 35.9\fbinv.

%\fi
