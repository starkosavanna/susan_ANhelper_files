\subsection{Electron Reconstruction and Identification}\label{eID}
%\textbf{Responsible: Z.Mao E.Laird}

Electrons are reconstructed by associating a track reconstructed in the silicon detector with a 
cluster of energy in the electromagnetic calorimeter (ECAL). One of the main challenges for electron 
reconstructions in the CMS detector arise from bremsstrahlung radiation in the silicon tracker. 
Additional to the energy loss of the electrons through radiation, radiated photons are spread  over 
several crystals of the ECAL detector along the electron trajectory, mostly in the $\phi$ direction  
(the magnetic field is in the z direction). 


To measure the initial energy of the electron accurately, two algorithms  based on energy clustering, 
``Hybrid'' for the barrel and  ``multi-5$\times$5'' for the endcaps, are used to measure the energy of 
electrons and photons. Electron tracks are reconstructed by matching hits in the silicon 
strip tracker to seed hits in the pixel  detector based on the combinatorial Kalman filter method. 
A pixel seed is composed of two or three pixel hits compatible with the beam spot. Once the hits are 
collected, a Gaussian Sum Filter (GSF) fit is performed to estimate the track parameters. For each GSF 
track, several PF clusters, corresponding to the electron at the ECAL surface and the bremsstrahlung 
photons emitted along its trajectory, are grouped together. Most of the bremsstrahlung photons are 
recovered in this way. In order to minimize the many possible trajectories due to different 
combinations of hits, the  track that best matches an energy supercluster in the ECAL is chosen to 
be the reconstructed track \cite{electron8TeV}.

The preselection of primary electron candidates requires  good geometrical matching and good
agreement between the  momentum of the track and the energy of the ECAL  supercluster. Three
quantities are used to estimate the geometrical matching: $\Delta \eta_{in} = \eta_{sc}
-\eta^{Track}_{vertex}$;  $\Delta \phi_{in} = \phi_{sc} -\phi^{Track}_{vertex}$ and a MVA technique 
that combines information on track observables, the electron PF cluster observables, and the 
association between the two . The $\eta_{sc}$ and $\phi_{sc}$ coordinates correspond to  the 
supercluster position and are measured using an energy weighted algorithm. The $\eta^{Track}_{vertex}$ 
and  $\phi^{Track}_{vertex}$ coordinates are the position of the track at the interaction vertex 
extrapolated,  as a perfect helix, to the ECAL detector. 

Electron selections have two main components, electron  identification and electron isolation. 
In this analysis we use the MVA electron identification. 
The MVA cuts used to define the 80\% and 90\% signal efficiency working points are summarized in Table~\ref{eIDtable}. 

\begin{table}[ht]
\begin{center}
 \caption{Electron ID Selections.\label{eIDtable}}
 \begin{tabular}{| l | c | c | c |}
 \hline\hline
       Category                             & MVA$_{\textrm{min cut}}$ (80\% signal eff)    & MVA$_{\textrm{min cut}}$ (90\% signal eff)  \\[0.5ex] \hline
       Barrel ($\eta < 0.8$) $p_{T}>10$     & 0.941                                         & 0.837             	\\
       Barrel ($\eta > 0.8$) $p_{T}>10$     & 0.899                                         & 0.715             	\\
       Endcap $p_{T}>10$                    & 0.758                                         & 0.357             	\\
 \hline
 \hline
 \end{tabular}
\end{center}
\end{table}

The electron relative combined isolation is defined as below:
\begin{equation}
\text{relative combined isolation}~= \frac{\Sigma p_T^{\text{charged had}} + \text{max}[0, \Sigma p_T^{\text{neutral had}} + 
\Sigma p_T^{\gamma} - 0.5\times\Sigma p_T^{\text{charged hadrons from PU}}]}{p_T^{\text{electron}}}
\end{equation}
where the sums run over the charged hadron candidates, neutral hadrons and photons, within a $\Delta R < 0.4$ around 
the electron direction. The charged hadron candidates are required to originate from the vertex of the event of 
interest, and $p_T^{\text{charged hadrons from PU}}$ is a correction related to event pileup.


In all channels, the identification and isolation used follows the POG recommended criteria. In order to improve the signal 
acceptance, especially at high mass,  the 90\% efficiency working point of the MVA electron ID is chosen.
%The electron trigger/identification efficiencies and scale factors used in these analyses have been taken from:
%``https://twiki.cern.ch/twiki/bin/view/CMS/MultivariateElectronIdentificationRun2"

