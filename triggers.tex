\section{Triggers}\label{sec:triggers}
%\textbf{Responsible: E. Laird, Z. Mao, A. Florez, F. Gonzalez} 

%We are using the triggers in Table~\ref{triggertable} for the analysis with 19.4 fb$^{-1}$ of data. 

For the \tlth final states, we use single lepton triggers instead of
$l\times\tau_{h}$ cross-triggers to maintain a similar strategy across
channels and also allow us to use the $\tau_{h}$ isolation sidebands
as control and validation samples. For the \tetm final state, any
trigger with iso-muon requirement, either a e-$\mu$ cross-trigger or
single-muon trigger, would eliminate the isolation sideband for QCD
estimation. Hence, we use the same single-electron trigger as for the
\teth final state. Trigger paths are summarized in
Table~\ref{tab:triggernames}. Note this constrains the object ID and
phase space that we are studying. For example, the use of these
triggers requires $p_{T}$ cuts of 35 GeV, 35 GeV, and 70 GeV for our
leading light leptons and $\tau_{h}$'s in the \tmth, \teth, and \thth
channels, respectively. We note that although the use of single lepton triggers instead of
$l\times\tau_{h}$ cross-triggers force us to increase the offline $p_{T}$ cuts, 
it doesn't affect our analysis since our signal is characterized by high-p$_{T}$ 
leptons. 
%Collision data events firing the {\bf HLT\_DoubleMediumIsoPFTau35\_Trk1\_v*} trigger are used for the 
Collision data events firing the {\bf HLT\_DoubleMediumIsoPFTau35\_Trk1\_eta2p1\_Reg*} trigger for Run BCDEFG and {\bf HLT\_DoubleMediumCombinedIsoPFTau35\_Trk1\_eta2p1\_Reg}  for Run H are used for the $\tau_{h}\tau_{h}$ channel. The offline-$p_{T}(\tau_{h})$ threshold used in this channel is driven by the trigger. The behaviour of this trigger as function of $p_{T}(\tau_{h})$ is explained in section~\ref{sec:tauTrigger}. 

\begin{table}[ht]
\begin{center}
  \caption{The trigger paths used to collected the data.  Emulated
    trigger paths, in particular those most similar to the paths used
    to collect the data, are applied to the simulated
    samples.\label{tab:triggernames}}
  \begin{tabular}{| l | c |}
  \hline
       Channel                      & Trigger Path                                   \\[0.5ex] \hline
       \thth (data, runs BCDEFG)    & HLT\_DoubleMediumIsoPFTau35\_Trk1\_eta2p1\_Reg \\
       \thth (data, run H)          & HLT\_DoubleMediumCombinedIsoPFTau35\_Trk1\_eta2p1\_Reg \\
       \tmth (data and MC)          & HLT\_IsoMu24                                   \\
       \teth, \tetm (data and MC)   & HLT\_Ele27\_eta2p1\_WPTight                    \\
  \hline
  \end{tabular}
\end{center}
\end{table}

\subsection{Single Lepton Trigger Efficiency}\label{sec:lepTrigger}
%\textbf{Responsible: E. Laird, Z. Mao} 


The single electron and muon trigger efficiencies are measured
using a tag and probe method where we selected events with at 
least one electron and one muon pair satisfying the following
requirements:

For electrons:
\begin{itemize}
  \item $p_T > 13~\gev$, $|\eta| < 2.1$, isolation $< 0.15$, $d_{xy}<0.045$~cm, $d_{z}<0.2$~cm
  \item passing electron MVAWP90 ID
  \item no matched conversions
  \item number of missing hits = 0
\end{itemize}

For muons:
\begin{itemize}
  \item $p_T > 10~\gev$, $|\eta| < 2.1$, isolation $< 0.15$, $d_{xy}<0.045$~cm, $d_{z}<0.2$~cm
  \item passing "Medium" muon ID
\end{itemize}

For the pair:
\begin{itemize}
  \item $\Delta$R(\tetm) $>$ 0.3
  \item choose the most isolated pair
  \item choose the opposite sign pair
  \item events are rejected if there is an additional electron with isolation $<$ 0.3, 
        or an additional muon with isolation $<$ 0.3.
\end{itemize}

\subsection{Single Electron Trigger Efficiency}\label{sec:eleTrigger}
%\textbf{Responsible: E. Laird, Z. Mao} 

To measure the single-electron trigger efficiency, after the
preselection mentioned above, we select (tag) events with a
single-muon trigger (HLT\_IsoMu24) with the offline muon, requiring $p_T > 24~\gev$,
 matching the HLT muon that fired the trigger.  Then, the trigger efficiency is
defined at the fraction of events which also pass (probe) the
single-electron trigger (HLT\_Ele27\_eta2p1\_WPTight).

The efficiency curves of the single-electron triggers, measured
vs. electron \pt, are shown in Figure~\ref{fig:eturnon}. A
$p_{T}>35$~\gev cut is motivated to avoid the turn-on region. The 
difference in trigger efficiency between data and simulation is 
measured in different detector sections: barrel ($|\eta| < 1.479$) 
and endcap ($1.479 < |\eta| < 2.1$). As shown, this difference has 
little dependence on the electron \pt. Thus, a flat line is fitted 
on this difference and we obtained a correction of 1.002 (0.957) 
for simulated events when the electron is in the barrel (endcap) 
region. Overall, we assign a 5\% systematic uncertainty for the 
single electron trigger efficiency corrections due to the outliers 
in the fit.


\begin{figure}\centering
  \includegraphics[width=0.45\textwidth]{figures/triggerStudy/eleTrigTurnOnCurve_ePt_em_WP90_barrel}
  \includegraphics[width=0.45\textwidth]{figures/triggerStudy/eleTrigTurnOnCurve_ePt_em_WP90_endcap}
  \caption{\label{fig:eturnon} The efficiency vs. \pt curves of the
    single-electron triggers used. Left column: $|\eta|<1.479$. Right column: $|\eta|>1.479$.}
\end{figure}

\subsection{Single Muon Trigger Efficiency}\label{sec:muTrigger}
%\textbf{Responsible: E. Laird, Z. Mao} 

To measure the single-muon trigger efficiency, after the preselection
mentioned above, we select (tag) events with a single-electron trigger 
(HLT\_Ele27\_eta2p1\_WPTight) with the offline electron, requiring 
$p_T > 28~\gev$, match the HLT electron that fired the trigger. Then, 
the trigger efficiency is defined at the fraction of events which also 
pass (probe) the single-muon trigger (HLT\_IsoMu24).

The efficiency curves of the single-muon triggers, measured vs. muon
\pt, are shown in Figure~\ref{fig:muturnon}. A $\pt>35$~\gev cut is
motivated to avoid the turn-on region and to be synchronized with the \teth 
channel. Similar to the single electron trigger, we measured the difference 
in trigger efficiency between data and simulated events in several regions: 
barrel ($|\eta| < 0.8$); middle ($0.8 < |\eta| < 1.24$); and endcap 
($1.24 < |\eta| < 2.1$). From the flat fit, the corrections for simulated 
events are found to be 0.973, 0.956 and 0.974 for events with muons in 
the $\eta$ regions mentioned above. We assign a 5\% (10\%) single muon trigger 
efficiency corrections systematics for muons in the central (endcap) region 
to accommodate the few outliers in the fit.

\begin{figure}\centering
  \includegraphics[width=0.45\textwidth]{figures/triggerStudy/muonTrigTurnOnCurve_mPt_em_WP90_barrel}
  \includegraphics[width=0.45\textwidth]{figures/triggerStudy/muonTrigTurnOnCurve_mPt_em_WP90_middle}
  \includegraphics[width=0.45\textwidth]{figures/triggerStudy/muonTrigTurnOnCurve_mPt_em_WP90_endcap}
  \caption{\label{fig:muturnon} The efficiency vs. \pt curves of the
    single-muon triggers used. Top left: $|\eta|<0.8$.  Top right: $0.8<|\eta|<1.24$. Bottom: $|\eta|>1.24$.}
\end{figure}


\subsection{Di-Tau Trigger Efficiency}\label{sec:tauTrigger}
%\textbf{Responsible:A. Florez, F. Gonzalez} 

The efficiency of the $\tau_{h}\tau_{h}$ trigger is measured using 
$Z\to\tau\tau\to\mu\tau_{h}$ events.  The $\tau_{h}$ candidates 
reconstructed in the selected $Z\to\tau\tau\to\mu\tau_{h}$ events are 
required to pass the same $\tau_{h}$ identification used for the final
analysis and which will be described in more detail in the sections to follow. The 
``denominator" selections used to define the $Z\to\tau\tau\to\mu\tau_{h}$ control sample 
are summarized below:

\begin{itemize}
  \item Events must fire the HLT{\_}IsoMu24 trigger 
  \item Exactly $1$ global $\mu$ with $|\eta| < 2.1, p_{T} > 24$ GeV
  \item ``isMediumMuon"
  \item Muon best track d$_{xy}$ $<$ 0.2~cm , d$_{z}$ $<$ 0.045~cm w.r.t. PV
  \item Relative $\mu$ isolation (with $\delta\beta$ corrections) $< 0.1$
  \item $\ge 1$ HPS $\tau_{h}$ with $|\eta| < 2.1, p_{T} > 20$ GeV
%  \item d$_{xy}$ w.r.t PV $<$ 0.2 , d$_{z}$ w.r.t PV $<$ 0.045 
  \item Muon veto: ``againstMuonTight3"
  \item Electron veto: ``againstElectronVLooseMVA6"
  \item Decay mode finding with 1 or 3 signal charged hadrons %(1, 2, or 3 signal charged hadrons; see section 5)
  \item Isolation: ``byTightIsoMVArun2v1DBnewDMwLT"
  \item $p_{T} >$ 5. GeV for leading track of $\tau_{h}$ with d$_{xy} <$ 0.2~cm and d$_{z} <$ 0.045~cm w.r.t. PV
  \item $\Delta R(\mu,\tau_{h}) > 0.5$
  \item $Q(\mu) \times Q(\tau_{h}) < 0 $
  \item $40 < m(\mu, \tau_{h}) < 80$ GeV
  \item $m_{T}(\mu,\MET) < 30$ GeV
  \item 0 jets tagged as b-jets
  \item 0 tagged electrons
\end{itemize}
The numerator is defined by additionally requiring those events to 
pass the \\ 
HLT{\_}IsoMu21{\_}eta2p1{\_}MediumIsoPFTau32{\_}Trk1{\_}eta2p1{\_}Reg{\_}v trigger.
The efficiency is measured for each $\tau_{h}$ leg individually and
parametrized as a function of $p_{T}$. 

\begin{figure}\centering
  \includegraphics[width=0.5\textwidth]{figures/triggerStudy/DiTauTriggerEfficiency_full_lumi.png}
  \caption{\label{fig:triggertau} The per-leg $\tau_{h}$ trigger efficiency as a function of p$_{T}(\tau_{h})$ for data obtained by requiring $\mu-\tau_{h}$ region.}
\end{figure}

The fit to the efficiency curve in Figure ~\ref{fig:triggertau} is performed with the following function:

\begin{equation}
pdf=P[0]+P[1]\frac{1}{\sqrt{2\pi}}\int_{-\infty}^{y}e^{-\frac{t^{2}}{2}}dt    
\end{equation}

where 

\begin{equation}
y=\frac{\sqrt{x}-\sqrt{P[2]}}{2P[3]}, \hspace{1cm} P[2]=\mu, P[3]=\sigma 
\end{equation}

We do not apply the trigger in MC, but instead model the correct per leg trigger efficiency observed in data by weighing the predictions from simulation using the fit of 
the trigger efficiency curve in data (black solid curve in the plot). The per-leg $\tau_{h}$ trigger efficiency is approximately 90\% at p$_{T}(\tau_{h})=60$ GeV and plateaus to about 95\% 
efficiency at 70 GeV. To be conservative, we assign a 5\% systematic uncertainty, per $\tau_{h}$ leg, on the trigger efficiency due to the ineffiency of the trigger at $p_{T} > 70$ GeV. 

\iffalse
\begin{table}[!htpb]
   \caption{Data and background yields in the $Z\to\tau\tau\to\mu\tau_{h}$ control sample used to measure the per leg trigger efficiency for the 
$\tau_{h}\tau_{h}$ trigger.}
   \centering{
    \begin{tabular}{| l | c |}
       \hline\hline
       Sample             & Events         \\ [0.5ex] \hline
       Data               &19578           \\
       t $\overline{t} $  &170.881$\pm$13.072          \\
       W+Jets           &2127.76$\pm$46.128          \\
       Z+Jets           &12218.5$\pm$110.544        \\
       QCD               &4003.93$\pm$63.277       \\
      \hline\hline
     \end{tabular}
   }        
   \label{table:BGYieldtable} % is used to refer this table in the text
 \end{table}
\fi
