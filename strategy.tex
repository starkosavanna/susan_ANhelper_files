\section{Strategy}\label{sec:strategy}
%% \textbf{Responsible: A. Gurrola, T. Kamon, F. Romeo, E. Laird} 
%\iffalse

A $\tau$ lepton is the heaviest known lepton with a mass of 1.777 GeV/$c^2$ and a lifetime of $2.9\times10^{-13}$ seconds. Around one third of
all $\tau$ leptons decay to $e/\mu$, and the remainder decay into hadronic jets ($\tau_{h}$). 
In the latter case, a $\tau_{h}$ consists of one, three, or (rarely) five charged mesons usually accompanied by one or
more neutral pions.

%As the $\tau$ lepton decays to $e\overline{\nu}_e}\nu_{\tau}~(17.8\%)$, $\mu\overline{\nu}_\mu}\nu_{\tau}~(17.4\%)$, and hadrons +
%$\nu_{\tau}~(64.8\%)$ we will refer to these decays as the $e$ decay channel, $\mu$ decay channel, and the  $\tau_\textrmr{h}$ decay
%channel. 

Four $\tau\tau$ final states, $\tau_\Pe \tau_{\mu}$, $\tau_\Pe \tauh$,
$\tau_{\mu} \tauh$, and \ditauhad, are selected, where $\tau_\ell$ ($\ell = \Pe,\mu$) and \tauh
refer to the leptonic and hadronic decay modes of the $\tau$ lepton, respectively.
%We consider four distinct analyses for pairs of $\tau$ lepton decays, namely \emu ~(6.2\%), \etau ~(23.1\%), \mutau  ~(22.5\%), and
%\ditauhad ~(42\%). 
We ignore the \EE and \MM channels due to the copious Drell-Yan
Z/$\gamma^*\rightarrow e^+e^-, \mu^+\mu^-$ production, although there is ongoing development of algorithms to discriminate prompt $e/\mu$ from $\tau$ lepton 
decays to light leptons which will be utilized with the 2016 data/analysis.

The overall strategy of the analysis is similar in all channels and analogous to previous iterations of this analysis with a few improvements which we point out below. 
In general, we identify events with two oppositely charged, nearly back--to--back objects. Because the $\tau\tau$ system decays with up
to four neutrinos, we expect to have missing transverse energy (\MET) present in the event. In contrast to \zprime searches in the
$e^+e^-$ and $\mu^+\mu^-$ channels, the visible \ditau mass does not produce a narrow peak due to the missing neutrinos.
Instead, we look for a broad enhancement in the \ditau$+$\MET invariant mass distribution consistent with new physics. The two main topological changes to the selections 
with respect to previous iterations of this analysis are: (1) the addition of a cut on the transverse mass between the $\tau_{\mu}$/$\tau_\Pe$ and $\MET$ in order to reduce the W+jets background 
in the $\tau_\Pe \tauh$ and $\tau_{\mu} \tauh$ channels. (2) the addition of a cut on the difference in azimuthal angle ($\Delta \phi$) between each of the tau legs and the $\MET$. This helps 
suppress SM backgrounds, while also providing one way to minimize the overlap of this analysis, which targets DY production of $Z^\prime$ bosons (i.e. $q\bar{q} \to Z^\prime$), with the ongoing 
VBF $Z^\prime$ analyses. As usual, our selections maintain high efficiency for signal events (order 10-20\%), provide strong background suppression, and reduce the influence of systematic
effects.  

As \zditau is both a background as well as an important validation for our signal process, our final selection requirements are such that by removing or
reversing just a few cuts we can obtain a relatively clean sample of \zditau events. We note that this is now possible also in the fully hadronic channel due 
to the improvements in the ditau trigger, allowing us to lower the p$_{T}$ thresholds on $\tauh$, and the improved $\tauh$ identification algorithms, which reduce the jet-$\tauh$ misidentification 
rate by about a factor of two. This also means the QCD multijet background yield is lower, particularly at high reconstructed ditau mass. The detailed trigger efficiency and $\tauh$ identitication 
efficiency/fake-rate studies are included in Sections 3--5. 

In order to ensure robustness of 
the analysis and our confidence in the results, whenever possible we rely on the data itself to understand and validate the efficiency of reconstruction
methods as well as the estimation of the background contributions. For that purpose we define control regions with most of the
selections similar to what we use in our main search but enriched with events from background processes. Once a background enhanced
region is created, we measure selection efficiencies in those regions and  extrapolate to the region  where we expect to observe our
signal. In cases where a complete data--driven method is not possible we make use of scale factors, ratio between observed data events
and expected MC events in the background enhanced region to validate or estimate the background  contribution in the signal region from simulation. Although each
individual channel could have its own set of requirements, we maintain, wherever possible, consistent definitions and selection
criteria  between channels. 

In addition to various plots for a given background enriched control region used to estimate its contribution 
to the signal region, we also provide ``control plots" to assess the level of agreement between data and MC. These control plots are 
typically produced using a set of ``preselections" where signal contamination is expected to be small and are also produced 
using final selections after unblinding the signal region (\textbf{we are currently blinded until given the green-light from the conveners}). 
In general, the DY and top pair contributions to these ``control plots" 
are taken directly from MC after separately validating, in their respective BG enriched control regions, that the data/MC scale factors are consistent 
with unity. The QCD shapes and yields are determined entirely from data. QCD accounts for roughly 90\% of the total source of background in the 
$\tau_{h} \tau_{h}$. The background is estimated using \MET side-bands and $\tau_{h}$-pairs with same and opposite electric charge. 
The W + jets shapes and yields in these control plots are also data-driven in 
in cases where the W + jets background is important (i.e. large contribution). In cases where the W + jets background is small in comparison to other backgrounds, 
the MC-based yields are corrected using data/MC scale factors obtained from the W + jets enriched control samples, while the shapes are taken entirely from simulation. 
%While the W + jets yields can be done in a data-driven way using scale factors, the shapes for various templates must be taken entirely from MC since 
%our data-driven background estimation methods are only constructed to extract the final mass templates, and not necessarily for other variables such as 
%$m_{T}(l,\MET)$, $p_{T}(\tau_{h})$, etc. 
We caution the reader that while these ``control plots" are useful to help guide the 
understanding of the analysis and data, in many cases it is not required that they show agreement (e.g. if certain control plots are dominated by MC-driven backgrounds or 
if certain parts of a distribution, such as the tails, are dominated by MC-driven backgrounds). 

To quantify the significance of any possible excess or set upper limits on the production rate, we perform a fit of the $m(\tau_{1},\tau_{2},\MET)$ mass 
distribution and employ the CL$_{s}$ technique to interpret the results in terms of the upper 95\% credibility level limits
for each channel. The joint limit is obtained by combining the posterior probability density functions (likelihood) and taking into
account correlation of systematic uncertainties within and across channels.

The structure of this note is as follows: Sections~\ref{sec:triggers} and~\ref{sec:samples} describes the data sets used in each analysis. 
Section~\ref{sec:recoID} provides a brief discussion of the reconstruction and identification of the objects used to reconstruct our  $\tau\tau$ 
pairs.
%, while Section~\ref{sec:commonSelection} describes the common selection across the four different channels.
Sections 6--\ref{sec:eMu} describe the specific selections and background extraction methods applied  to each
individual channel. In Sections~\ref{sec:systematics}--\ref{sec:results} we describe the statistical method used to extract 
the $95\%$ C.L. upper limits and the results of the analysis. 
%Finally, Section~\ref{sec:conclusions} summarizes the results and 
%overall sensitivity.

%\fi
