\section{Electron + Hadronic Tau Channel}\label{sec:eTauhad}
%\textbf{Responsible: E. Laird, Z. Mao, U. Heintz}

\subsection{Event selection}\label{sec:et_selection}

Events must fire the single-electron trigger described in
Section~\ref{sec:triggers}.  We select reconstructed electrons
satisfying:
\begin{itemize}
  \item $p_T>35~\gev$ and $\vert\eta\vert<2.1$;
  \item distance of closest approach to the leading 
sum-$p_T^2$ primary vertex of less than 0.045~cm 
(transvese) and 0.2~cm (longitudinal);
  \item passing the loose working point of the e/$\gamma$
POG MVA ID;
  \item having no matched conversion nor missing hits;
  \item within $\Delta R<0.5$ of the HLT electron that fired the trigger.

\end{itemize}

Offline $\tauh$'s are required to have:
\begin{itemize}
  \item $p_{T}> 20~\gev$ and $\vert \eta \vert < 2.1$;
  \item distance of closest approach to the leading;
sum-$p_T^2$ primary vertex of less than 0.2~cm (longitudinal)
  \item pass the new Decay Mode Finding requirement as 
either a 1-prong or 3-prong $\tauh$;
  \item pass the ``againstElectronTightMVA6'' and
``againstMuonLoose3'' identification requirements.
\end{itemize}

We build pairs of electrons and $\tauh$'s in which the electron and
$\tauh$ are separated by at least $\Delta R > 0.5$.  In events with
more than one such pair, we select the pair with the two most isolated
leptons, considering first the electron, and then the $\tauh$.  This
criterion was seen to have good efficiency for signal samples.  In the
rare case of multiple such pairs having identical isolation values,
the reconstructed $p_T$'s are considred, preferring higher values.

After a pair has been chosen for an event, we apply the following
isolation requirements on the leptons, for an event to enter the
signal region: electron relative isolation $<0.15$; \tauh isolation
``byTightIsolationMVArun2v1DBnewDMwLT.''  In order to keep the
different final states exclusive, an event is rejected if there is an
additional electron satisfying the above identification requirements
and with relative isolation $<0.3$, or a muon satisfying the
identification requirements described in
Section~\ref{sec:em_selection} with relative isolation $<0.3$.  To
reduce further possible di-electron events in the $\teth$ channel, an
event is rejected if there is an opposite-charge electron pair with
$\Delta R > 0.15$ in which both of the electrons satisfy $p_T >
15~\gev$, $\vert \eta \vert < 2.5$, the e/$\gamma$ POG ``veto'' ID,
and relative isolation $<0.3$. To further reduce backgrounds, we require 
the electron and $\tauh$ to have opposite charge. The selection set 
mentioned above is defined as preselection.

Following the preselection, as for the $\tmth$ channel, the signal 
region is defined as having
\begin{itemize}
  \item $\cos{\Delta \phi (e,\tauh)}<-0.95$;
  \item $\ETslash>30~\gev$;
  \item $\cos{\Delta \phi (e,\ETslash)} > 0.9~\text{or}~(\cos{\Delta \phi (\tauh,\ETslash)} > 0.9~\text{and}~m_{\text{T}}(e,\ETslash) > 150~\gev)$;
  \item no jet with $p_T>30\gev$ tagged as a b-jet (CSV loose).
\end{itemize}

The distributions of these variables after preselection, and after
selection requirements on the other variables, are shown in
Figure~\ref{fig:et_nm1_distributions}.
\begin{figure}\centering
  \includegraphics[width=0.31\textwidth]{figures/n_1/et/met_et}
  \includegraphics[width=0.31\textwidth]{figures/n_1/et/cosDPhi_et}
  \includegraphics[width=0.31\textwidth]{figures/n_1/et/nb_et}
  \includegraphics[width=0.31\textwidth,page=1]{figures/n_1/et/cosDPhi_lep_met_et}
  \includegraphics[width=0.31\textwidth,page=2]{figures/n_1/et/cosDPhi_lep_met_et}

  \caption{\label{fig:et_nm1_distributions} (Data driven QCD with only statistical uncertainties) 
Distributions of the variables used for \teth signal selection, after all other signal selection requirements 
on variables other than the one plotted. Top row plots are made with data driven W+Jets in distributions of: 
\ETslash (top left), $\cos{\Delta \phi (e, \tauh)}$ (top middle, with $\cos{\Delta \phi (e, \tauh)} > 0$) 
and $n_b$ (top right). Bottom row plots are made with MC W+Jets in distributions of: 
$\cos{\Delta \phi (e, \ETslash)}$ (bottom left) and $\cos{\Delta \phi (\tauh, \ETslash)}$ (bottom right).}
\end{figure}

The Standard Model processes considered as backgrounds are Drell-Yan,
di-boson production, top quark single and pair production, W+jets
production, and QCD multi-jet production.

\subsection{Genuine dilepton events}

Studies of simulated events indicate that for Drell-Yan process, top
quark single and pair production, and di-boson production, the
reconstructed and selected electrons and hadronic taus are typically
associated with genuine simulated leptons.  The nominal expected event
rates are estimated by scaling the simulated samples by the best
available cross sections, listed in Table~\ref{tab:mc_samples}, and by
the integrated luminosity of the data samples.

\subsubsection{Drell-Yan process}
Due to large W+Jets and QCD contamination, as shown in the left panel of 
Figure~\ref{fig:et_dy_tt} with the following selections:
\begin{itemize}
  \item $\ETslash<30~\gev$;
  \item no jet with $p_T>30\gev$ tagged as a b-jet (CSV loose);
  \item $\cos{\Delta \phi (e,\ETslash)} > 0.9~\text{or}~\cos{\Delta \phi (\tauh,\ETslash)} > 0.9$;
  \item 60 GeV $<$ \meffetau $<$ 150 GeV,
\end{itemize}
we use the Drell-Yan rate systematic uncertainty (5\%) estimated from
the \tetm final state in~\ref{sec:em_DY} for its higher Drell-Yan 
purity. However, we cross check this here and compare the Drell-Yan 
rate between data and MC:
\begin{equation}\label{eq:et_DY}
\frac{\text{data - non-Drell-Yan backgrounds}}{\text{MC Drell-Yan}} = 0.95 \pm 0.05\%\quad,
\end{equation}
which agrees with the measurement in in~\ref{sec:em_DY}. 


\subsubsection{$t\bar{t}$ and single top processes process}
For \teth, we estimate the \ttbar + single top production rate
systematics (1\%), with the top pt reweighting as described in 
~\ref{sec:mt_tt}, from \tetm final state, as described in
 ~\ref{sec:em_tt}, for its higher top purity.  However 
we cross check this in a \teth top-rich region, defined by 
the following selections and shown in the right panel of 
Figure~\ref{fig:et_dy_tt}:
\begin{itemize}
  \item $\cos{\Delta \phi (e,\tau_{h})}<-0.95$;
  \item $\ETslash>30~\gev$;
  \item $\cos{\Delta \phi (e,\ETslash)} > 0.9~\text{or}~(\cos{\Delta \phi (\tau_{h},\ETslash)} > 0.9~\text{and}~m_{\text{T}}(e,\ETslash) > 150~\gev)$;
  \item at least one jet with $p_T>30\gev$ tagged as a b-jet (CSV loose).
\end{itemize}
The $t\bar{t}$ + single top data/MC overall agreement is estimated to be:
\begin{equation}\label{eq:et_tt}
\frac{\text{data - non $t\bar{t}$ + single top backgrounds}}{\text{MC $t\bar{t}$ + single top}}  =  0.96 \pm 0.06 \quad,
\end{equation}
which agrees with the measurement in in~\ref{sec:em_DY}. 

\begin{figure}\centering
  \includegraphics[width=0.45\textwidth]{figures/ControlRegions/DY_check_et}
  \includegraphics[width=0.45\textwidth]{figures/ControlRegions/ttbar_check_et}
  \caption{\label{fig:et_dy_tt}(Data driven W+Jets and QCD with only statistical uncertainties) Distributions of \meffetau. Left:
    validation region with $\ETslash<30~\gev$, $n_b = 0$ and $60 \gev<$  \meffetau $<$ 150 GeV.  Right:
    validation region with $n_b\geq1$.}
\end{figure}

\subsubsection{Di-boson process}
Di-boson processes are a relatively small background in the electron + 
hadronic tau channel. They are estimated directly from simulation with 
a 20\% systematic uncertainty measured from the electron + muon channel 
in section \ref{section:diBoson}.
    
\subsection{QCD multi-jet background}\label{sec:etau_qcd}
Similar as  the \tmth channel, we construct a data-driven template for
the shape of the QCD multi-jet background, i.e. the processes lacking
prompt leptons.  Based on the charge of the final state, the electron 
relative isolation and \tauh isolation, we split the events into 
four regions shown in Figure~\ref{fig:ABCDEF} and described below:
\begin{itemize}
  \item A (Signal) Region: e and \tauh have opposite charge, e relative isolation is less than $0.15$, and \tauh pass "Tight" isolation requirement.
  \item B Region: e and \tauh have same charge, e relative isolation is less than $0.15$, and \tauh pass "Tight" isolation requirement.
  \item E Region: e and \tauh have opposite charge, e relative isolation is between $0.15$ and $1.0$, and \tauh fail "Tight" isolation requirement.
  \item F Region: e and \tauh have same charge, e relative isolation is between $0.15$ and $1.0$, and \tauh fail "Tight" isolation requirement.
\end{itemize}

Table \ref{table:SF_QCD_et} shows the yields of data and MC samples in regions B 
and F used for the calculation of $f_\mathrm{LT}^\mathrm{QCD}$.

{\renewcommand{\arraystretch}{1.3}%
\begin{table}
   \caption{\label{table:SF_QCD_et} Event yields in regions B and F used for the calculation of $f_\mathrm{LT}^\mathrm{QCD}$.}
   \centering{
     \begin{tabular}{ | l | c | c | }
        \hline \hline
        Process                         & region B      & region F      \\ \hline
        Z + jets                        & 151 $\pm$ 26  & 0 $\pm$ 0     \\ \hline
        $t\bar{t}$                      & 40 $\pm$ 4    & 2.0 $\pm$ 0.8 \\ \hline
        W + jets                        & 418 $\pm$ 94  & 3 $\pm$ 2     \\ \hline
        DiBoson                         & 15 $\pm$ 2    & 0.1 $\pm$ 0.0 \\ \hline
        non-QCD background              & 624 $\pm$ 98  & 5 $\pm$ 2     \\ \hline
        Data                            & 2265          & 1539          \\ \hline \hline
        Data - non-QCD background       & 1641 $\pm$ 98 & 1539          \\ \hline
        $f_\mathrm{LT}^\mathrm{QCD}$    &  \multicolumn{2}{c|}{1.07 $\pm$ 0.08} \\ \hline
     \end{tabular}
   }
 \end{table}}

To check whether if the QCD shape  in the anti-isolated region correctly 
models the QCD shape in the isolated region, we compare the 
observation and background estimation in region B with the QCD shape 
taking from region F and normalized to the QCD in region B.  
An example of this test is shown in the left panel of 
Fig.~\ref{fig:et_dataDrivenChecks}. Here, W+jets 
is estimated directly from MC simulations. Overall, the data and 
estimated background agrees reasonably. Due to low statistics of 
the W+jets MC sample, some bins disagree by 20\%.

\begin{figure}\centering
  \includegraphics[width=0.45\textwidth]{figures/backgroundEstimation/et/regionB_et}
  \includegraphics[width=0.45\textwidth]{figures/backgroundEstimation/et/WJets_SF_et}
  \caption{\label{fig:et_dataDrivenChecks}(Only statistical uncertainties are included) Distributions of \meffetau. Left:
    in region B with signal region selections (data driven QCD and MC based W+jets).  Right:
    in region A' with signal region like selections (data driven QCD and W+jets).}
\end{figure}

\subsection{W+jets background}
\label{sec:et_w_bkg_validation}

Similar as done in the \tmth channel,  we construct 
a data-driven template for the shape of the W+jets background in a control 
region with high W+jets purity. Based on the charge of the final state and 
\tauh isolation, we split the events into four regions shown in 
Fig.~\ref{fig:ABCDEF} and described below:
\begin{itemize}
  \item A (Signal) Region: e and \tauh have opposite charge, e relative isolation is less than $0.15$, and \tauh pass "Tight" isolation requirement.
  \item B Region: e and \tauh have same charge, e relative isolation is less than $0.15$, and \tauh pass "Tight" isolation requirement.
  \item C Region: e and \tauh have opposite charge, e relative isolation is less than $0.15$, and \tauh pass "VeryLoose" isolation requirement but fail "Tight" isolation requirement.
  \item D Region: e and \tauh have same charge, e relative isolation is less than $0.15$, and \tauh pass "VeryLoose" isolation requirement but fail "Tight" isolation requirement.
\end{itemize}

In region C, W+jets events are estimated by subtracting non W+jets events
with genuine leptons (estimated by simulation) and data-driven QCD events 
(estimated from region E with a scale factor measured between D and F) 
bin-by-bin from data. W+jets events in the signal region most likely 
consists of one genuine lepton (electron in this channel) with a jet 
faking a \tauh. Thus, W+jets event in the signal region can be estimated 
by applying a "jet to tau fake rate" to the estimated W+jets events in region C.

To properly estimate the most relevant (closest to our signal region as possible) 
"jet to tau fake rate" from data, we construct a W+jets rich region with the 
following selections:
\begin{itemize}
  \item $\cos{\Delta \phi (e,\tau_{h})}<-0.95$
  \item $\ETslash>30~\gev$
  \item $0.5 < \cos{\Delta \phi (e,\ETslash)} < 0.9~\text{and}~55~\gev~< m_{\text{T}}(e,\ETslash) < 120~\gev$
  \item no jet with $p_T>30\gev$ tagged as a b-jet (CSV loose)
\end{itemize}
The "jet to tau fake rate" is estimated the same way as in Eq.~\ref{eq:fakeRate}.
Region A' is shown in the right panel of Fig.~\ref{fig:et_dataDrivenChecks} 
with the "jet to tau fake rate" estimated as: $f_{\text{jet}\rightarrow\tau} = 0.20 \pm 0.03$.

\subsection{Validation of Background Estimations}
Additional to the validation tests shown in Fig.~\ref{fig:et_dy_tt}, a test 
to simultaneously check the data driven QCD and W+jets estimations is performed 
by reverting the $\cos{\Delta \phi (e,\tauh)}$ cut in the following configuration:
\begin{itemize}
  \item $ -0.95 < \cos{\Delta \phi (e,\tauh)} < 0$
  \item $\ETslash>30~\gev$
  \item $\cos{\Delta \phi (e,\ETslash)} > 0.9~\text{or}~(\cos{\Delta \phi (\tauh,\ETslash)} > 0.9~\text{and}~m_{\text{T}}(e,\ETslash) > 150~\gev)$
  \item no jet with $p_T>30\gev$ tagged as a b-jet (CSV loose)
\end{itemize}

Figure~\ref{fig:revertCosDPhi_et} shows the distributions of \meffetau, \ETslash, 
electron \pt and \tauh \pt in region A with the above set of selections. 

\begin{figure}\centering
  \includegraphics[width=0.45\textwidth,page=3]{figures/ControlRegions/revertCosDPhi_et}
  \includegraphics[width=0.45\textwidth,page=4]{figures/ControlRegions/revertCosDPhi_et}\\
  \includegraphics[width=0.45\textwidth,page=1]{figures/ControlRegions/revertCosDPhi_et}
  \includegraphics[width=0.45\textwidth,page=2]{figures/ControlRegions/revertCosDPhi_et}
  \caption{\label{fig:revertCosDPhi_et} (Data driven W+Jets and QCD with only statistical uncertainties)
 Distributions with $-0.95 < \cos{\Delta \phi (e,\tauh)} < 0$ selection. Top: left: 
\meffetau right: \ETslash. Bottom: left: e \pt right: \tauh \pt.}
\end{figure}

%%Figure~\ref{fig:signalRegion_et} shows the distributions of \meffmtau in 
Figure~\ref{fig:SignalRegionPlot_a} shows the distributions of \meffmtau in 
the signal region.% with data blinded.

\begin{figure}\centering
  %% \includegraphics[width=0.6\textwidth,page=1]{figures/signalRegion/et_blind}
  \includegraphics[width=0.6\textwidth,page=1]{figures/signalRegion/et}
  \caption{\label{fig:signalRegion_et} (Data driven W+Jets and QCD with only statistical uncertainties)
 \meffmtau distribution with signal region selections.}
\end{figure}


\iffalse

\begin{figure}\centering
  \includegraphics[width=0.45\textwidth]{figures/et-em/Data_MC_Comparison/et_puweight_OS_signalRegionReversePZeta_1and3prong_lessThan-0p8}
  \caption{\label{fig:et_meff_flipped_pz} \teth channel: the
    distribution of reconstructed parent mass, \meffetau, in the
    W-validation region described in
    Section~\ref{sec:et_w_bkg_validation}.}
\end{figure}

\clearpage
\subsection{Overlays of observations and SM predictions}
\label{sec:et_overlays}

The expected SM event yields in the signal region are shown in
Figure~\ref{fig:etau_sm_template_and_mt}.
\begin{figure}\centering
  \includegraphics[width=0.45\textwidth]{figures/et-em/eTauStudies/et_puweight_OS_signalRegion_MCWJets_1and3prong}
  \includegraphics[width=0.45\textwidth]{figures/et-em/kinematicDistributions_unblind/et_puweight_OS_signalRegion_1and3prong_mt}
  \caption{\label{fig:etau_sm_template_and_mt} Left: predicted
    background yields and observed event yields in the \teth channel
    after signal selection.  Right: the distribution of transverse
    mass.}
\end{figure}

Distributions of \pt and $\eta$ are shown in Figure~\ref{fig:etau_sr_pt_eta}.
\begin{figure}\centering
  \includegraphics[width=0.45\textwidth]{figures/et-em/kinematicDistributions_unblind/et_puweight_OS_signalRegion_1and3prong_ePt}
  \includegraphics[width=0.45\textwidth]{figures/et-em/kinematicDistributions_unblind/et_puweight_OS_signalRegion_1and3prong_eEta} \\
  \includegraphics[width=0.45\textwidth]{figures/et-em/kinematicDistributions_unblind/et_puweight_OS_signalRegion_1and3prong_tPt}
  \includegraphics[width=0.45\textwidth]{figures/et-em/kinematicDistributions_unblind/et_puweight_OS_signalRegion_1and3prong_tEta}
  \caption{\label{fig:etau_sr_pt_eta} Distributions, after \teth final
    selection, of electron \pt (top left), electron pseudo-rapidity
    (top right), $\tau$ \pt (bottom left), $\tau$ pseudo-rapidity
    (bottom right).}
\end{figure}

\fi

