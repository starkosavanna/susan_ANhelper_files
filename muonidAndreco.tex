\subsection{Muon Reconstruction and Identification}\label{ss:muonreco}
%\textbf{Responsible: Z.Mao E.Laird}

Muon reconstruction is a multistep process that begins with the information gathered from the muon subdetector. As a
first step, standalone muons are reconstructed from hits in the individual drift tube (DT) and
cathode strip (CSC) chambers. Hits from the innermost muon stations are combined with hits in the other muon segments
using the Kalman-filter technique. The standalone muon trajectory is reconstructed by extrapolating from the
innermost muon station to the outer tracker surface resulting muon candidates are called ``standalone muons''. 
This standalone trajectory is then used to find a matching track reconstructed in the inner silicon tracker resulting 
in a ``global'' muon. Muon reconstruction is described in more detail in \cite{CMS_MUO_10-004}.

%Once a muon is required to have matching tracks in the inner and outer detectors,
%the main source of background consists of charged hadrons that leave a
%signature in the inner silicon tracker while also penetrating through the hadronic calorimeter and creating hits in the
%muon chambers. Charged hadrons that penetrate the hadronic calorimeter and leave hits in the muon system will deposit 
%significant energy in the calorimeters. Therefore, a calorimeter compatibility algorithm can be used to significantly 
%reduce the number of charged pion fakes. However, calorimeter compatibility is not used in this analysis due to our 
%uncertainty of the performance of such algorithms in the presence of high PU. 

The presence of punch-throughs, when a charged hadron penetrates through the hadronic calorimeter and creating hits 
in the muon chambers, often occur due to pions from the fragmentation of quarks and gluons. These punch-throughs 
can often be discriminated against by requiring isolation. Similarly, non-prompt muons from heavy flavor decays and 
decays in flights are expected to be within jets and can be discriminated against by imposing an isolation requirement. 
Muon identification is described in more detail in \cite{CMS_MUO_10-004} and \cite{CMS_MUO_11-001}.

Isolated muons are required to have minimal energy from PF neutral and charged candidates in a cone of $\Delta R =
0.4$ around the muon trajectory. PF charged candidates considered in the calculation of isolation are required to be near the 
primary vertex. Similar to electrons, relative isolation for muons is defined as:

\begin{equation}
\text{relative combined isolation}~= \frac{\Sigma p_T^{\text{charged had}} + \text{max}[0, \Sigma p_T^{\text{neutral had}} + 
\Sigma p_T^{\gamma} - 0.5\times\Sigma p_T^{\text{charged hadrons from PU}}]}{p_T^{\text{muon}}}
\label{eq:muIso}
\end{equation}
where the sums run over the charged hadron candidates, neutral hadrons and photons, within a $\Delta R < 0.4$ around 
the muon direction. The charged hadron candidates are required to originate from the vertex of the event of 
interest, and $p_T^{\text{charged hadrons from PU}}$ is a correction related to event pileup.

Table~\ref{table:muonidcuts} shows the complete list of cuts for the ``isMedium" $\mu$ identification criteria used in this analysis.
In all channels, the identification and isolation used follows the POG recommended criteria. 

\begin{table}[ht]
  \caption{$\mu$ Identification}
  \centering{
  \begin{tabular}{| l | c |}
  \hline\hline
        Cut \\ [0.5ex] \hline
        $\textrm{muon::isLooseMuon(recoMu)}$ \\
        $\textrm{recoMu.innerTrack()-}$$>\textrm{validFraction()}$$> 0.49$ \\
        AND \\
        $\textrm{recoMu.isGlobalMuon()}$\\
        $\textrm{recoMu.globalTrack()-}$$>\textrm{normalizedChi2()}$$< 3$ \\
        $\textrm{recoMu.combinedQuality().chi2LocalPosition}$$< 12$ \\
        $\textrm{recoMu.combinedQuality().trkKink}$$< 20$ \\
        $\textrm{muon::segmentCompatibility(recoMu)}$$> 0.303$ \\
        OR \\
        $\textrm{muon::segmentCompatibility(recoMu)}$$> 0.451$ \\
  \hline
  \hline
  \end{tabular}
  }
  \label{table:muonidcuts} % is used to refer this table in the text
\end{table}
