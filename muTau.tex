\section{Muon + Hadronic Tau Channel}\label{sec:mTauhad}
%\textbf{Responsible: E. Laird, Z. Mao, U. Heintz}

\subsection{Event selection}\label{sec:mt_selection}

Events must fire the single-muon trigger described in
Section~\ref{sec:triggers}.  We select reconstructed muons
satisfying:
\begin{itemize}
  \item $p_T>35~\gev$ and $\vert\eta\vert<2.1$
  \item distance of closest approach to the leading 
sum-$p_T^2$ primary vertex of less than 0.045~cm 
(transvese) and 0.2~cm (longitudinal)
  \item satisfy the muon POG medium muon requirement
  \item having no matched conversion nor missing hits
  \item within $\Delta R<0.5$ of the HLT muon that fired the trigger

\end{itemize}

Offline $\tauh$'s are required to have:
\begin{itemize}
  \item $p_{T}> 20~\gev$ and $\vert \eta \vert < 2.1$
  \item distance of closest approach to the leading
sum-$p_T^2$ primary vertex of less than 0.2~cm (longitudinal)
  \item pass the new Decay Mode Finding requirement as 
either a 1-prong or 3-prong $\tauh$
  \item pass the ``againstElectronVLooseMVA6'' and
``againstMuonTight3'' identification requirements
\end{itemize}

We build pairs of muons and $\tauh$'s in which the muon and
$\tauh$ are separated by at least $\Delta R > 0.5$.  In events with
more than one such pair, we select the pair with the two most isolated
leptons, considering first the muon, and then the $\tauh$.  This
criterion was seen to have good efficiency for signal samples.  In the
rare case of multiple such pairs having identical isolation values,
the reconstructed $p_T$'s are considred, preferring higher values.

After a pair has been chosen for an event, we apply the following
isolation requirements on the leptons, for an event to enter the
signal region: muon relative isolation $<0.15$; \tauh isolation
``byTightIsolationMVArun2v1DBnewDMwLT.''  In order to keep the
different final states exclusive, an event is rejected if there is an
additional muons satisfying the above identification requirements
and with relative isolation $<0.3$, or a electron satisfying the
identification requirements described in
Section~\ref{sec:em_selection} with relative isolation $<0.3$.  To
reduce further possible di-muon events in the $\tmth$ channel, an
event is rejected if there is an opposite-charge muon pair with
$\Delta R > 0.15$ in which both of the muon satisfy $p_T >
15~\gev$, $\vert \eta \vert < 2.5$, has hits both in the tracker and muon chambers,
and relative isolation $<0.3$. To further reduce backgrounds, we require 
the muon and $\tauh$ to have opposite charge. The selection set 
mentioned above is defined as preselection.

\subsection{Signal region selections}\label{sec:signalRegionSelection}
The topology of a massive $Z'\to\tau\tau$ decay 
consists of two energetic back-to-back going $\tau$'s. As $\tau$'s 
decay in the detector, undetectable neutrinos are produced leaving 
imbalance in the total transverse momentum. Backgrounds without 
such a topology are reduced by requiring $\ETslash>30 \gev$ and 
$\cos{\Delta \phi (\mu,\tauh)}<-0.95$. To further reduce $t\bar{t}$ 
backgrounds, events with one or more jets passing the CSV loose 
b-tag requirements are rejected from the signal region.

W+jets is one of the most dominant backgrounds in the \ltau channels, 
where the lepton comes from a leptonically decaying W and a 
random jet fakes the \tauh. The neutrino from the W decay is likely 
to be back-to-back with respect to the lepton. With the 
$\cos{\Delta \phi (\mu,\tauh)}<-0.95$ requirement, the \ETslash, 
due to neutrinos from W decays, are most likely in the same 
direction as \tauh. However, for $Z'\to\tau\tau$ decays, two 
neutrinos are expected in the lepton direction while only one 
neutrino is expected in the \tauh direction. Thus, the \ETslash 
in $Z'\to\tau\tau$ decays is more likely to be aligned with the 
reconstructed lepton. Figure~\ref{fig:selectionMotiviation_cosDPhi} 
shows, with the $\ETslash>30 \gev$, $\cos{\Delta \phi (\mu,\tauh)}<-0.95$ 
and 0 b-jet requirements, the separation in $\phi$ between the muon 
and \ETslash in the left panel and \tauh and \ETslash in the right panel.

\begin{figure}\centering
  \includegraphics[width=0.45\textwidth,page=1]{figures/n_1/mt/cosDPhi_lep_met_mt}
  \includegraphics[width=0.45\textwidth,page=2]{figures/n_1/mt/cosDPhi_lep_met_mt}
  \caption{\label{fig:selectionMotiviation_cosDPhi} (Data driven QCD and MC based W+Jets with only statistical uncertainties) 
Left: Distribution of $\cos{\Delta \phi (\mu,\ETslash)}$. Right: Distribution of 
$\cos{\Delta \phi (\tauh,\ETslash)}$}
\end{figure}

As shown, most of the W+jets background can be rejected by 
$\cos{\Delta \phi (\mu,\ETslash)} > 0.9$. However, there are roughly
20\% of the signal events in the $\cos{\Delta \phi (\tauh,\ETslash)} > 0.9$ 
category. Figure~\ref{fig:selectionMotiviation_mT} shows the 
$m_{\text{T}}(\mu,\ETslash)$ distribution of events, whose \ETslash is 
aligned with \tauh. By requiring $m_{\text{T}}(\mu,\ETslash) > 150~\gev$, 
most of the W+jets background will be rejected while having little 
to no effect on signal events. 

\begin{figure}\centering
  \includegraphics[width=0.45\textwidth,page=1]{figures/selections/mT_mt}
  \includegraphics[width=0.45\textwidth,page=2]{figures/selections/mT_mt}
  \caption{\label{fig:selectionMotiviation_mT} With $\cos{\Delta \phi (\tauh,\ETslash)} > 0.9$ 
selection. Left: (Data driven QCD and MC based W+Jets with only statistical uncertainties) 
Distribution of $m_{\text{T}}(\mu,\ETslash)$ with Z'(3000). Right: 
$m_{\text{T}}(\mu,\ETslash)$ of Z'(750), Z'(1750) and Z'(3000)}
\end{figure}
 

Hence, following the preselection, the signal region is defined as having:
\begin{itemize}
  \item $\cos{\Delta \phi (\mu,\tauh)}<-0.95$;
  \item $\ETslash>30~\gev$;
  \item $\cos{\Delta \phi (\mu,\ETslash)} > 0.9~\text{or}~(\cos{\Delta \phi (\tauh,\ETslash)} > 0.9~\text{and}~m_{\text{T}}(\mu,\ETslash) > 150~\gev)$;
  \item no jet with $p_T>30\gev$ tagged as a b-jet (CSV loose)\quad.
\end{itemize}

The distributions of these variables after preselection, and after
selection requirements on the other variables, are shown in
Figure~\ref{fig:mt_nm1_distributions}.
\begin{figure}\centering
  \includegraphics[width=0.31\textwidth]{figures/n_1/mt/met_mt}
  \includegraphics[width=0.31\textwidth]{figures/n_1/mt/cosDPhi_mt}
  \includegraphics[width=0.31\textwidth]{figures/n_1/mt/nb_mt}
  \caption{\label{fig:mt_nm1_distributions} (Data driven W+Jets and QCD with only statistical uncertainties) 
Distributions of the variables used for \tmth signal selection, after all other signal 
selection requirements on variables other than the one plotted:
    \ETslash (left), $\cos{\Delta \phi (\mu,\tauh)}$ (middle, with $\cos{\Delta \phi (\mu, \tauh)} > 0$) and $n_b$ (right).}
\end{figure}

The Standard Model processes considered as backgrounds are Drell-Yan,
di-boson production, top quark single and pair production, W+jets
production, and QCD multi-jet production.

\subsection{Genuine dilepton events}

Studies of simulated events indicate that for Drell-Yan process, top
quark single and pair production, and di-boson production, the
reconstructed and selected electrons and hadronic taus are typically
associated with genuine simulated leptons.  The nominal expected event
rates are estimated by scaling the simulated samples by the best
available cross sections, listed in Table~\ref{tab:mc_samples}, and by
the integrated luminosity of the data samples.

\subsubsection{Drell-Yan process}
Due to large W+Jets and QCD contamination, as shown in the left panel of 
Figure~\ref{fig:mt_dy_tt} with the following selections:
\begin{itemize}
  \item $\ETslash<30~\gev$;
  \item no jet with $p_T>30\gev$ tagged as a b-jet (CSV loose);
  \item $\cos{\Delta \phi (\mu,\ETslash)} > 0.9~\text{or}~\cos{\Delta \phi (\tauh,\ETslash)} > 0.9$;
  \item 60 GeV $<$ \meffmtau $<$ 150 GeV\quad,
\end{itemize}

we use the Drell-Yan rate systematic uncertainty (5\%) estimated from
the \tetm final state in~\ref{sec:em_DY} for its higher Drell-Yan 
purity. However, we cross check this here and compare the Drell-Yan 
rate between data and MC:
\begin{equation}\label{eq:mt_DY}
\frac{\text{data - non-Drell-Yan backgrounds}}{\text{MC Drell-Yan}} = 1.04 \pm 0.03 \quad,
\end{equation}
which agrees with the measurement in in~\ref{sec:em_DY}.

\subsubsection{$t\bar{t}$ and single top processes process}\label{sec:mt_tt}
For \tmth, we estimate the \ttbar + single top production rate
systematics (1\%), from  \tetm final state, as described in 
~\ref{sec:em_tt}, for its higher top purity.  To accommodate 
the difference between data and theoretical calculations in \ttbar 
differential cross section, \ttbar events are reweighted based 
on the momentum of the top quarks and the latest scale factors 
recommended by the Top POG.

However we cross check this in a \tmth top-rich region, 
defined by the following selections and shown in the right 
panel of Figure~\ref{fig:mt_dy_tt}:
\begin{itemize}
  \item $\cos{\Delta \phi (\mu,\tau_{h})}<-0.95$
  \item $\ETslash>30~\gev$
  \item $\cos{\Delta \phi (\mu,\ETslash)} > 0.9~\text{or}~(\cos{\Delta \phi (\tau_{h},\ETslash)} > 0.9~\text{and}~m_{\text{T}}(\mu,\ETslash) > 150~\gev)$
  \item at least one jet with $p_T>30\gev$ tagged as a b-jet (CSV loose)
\end{itemize}
The $t\bar{t}$ + single top data/MC overall agreement is estimated to be:
\begin{equation}\label{eq:mt_tt}
\frac{\text{data - non $t\bar{t}$ + single top backgrounds}}{\text{MC $t\bar{t}$ + single top}}  =  1.02 \pm 0.04 \quad,
\end{equation}
which agrees with the measurement in in~\ref{sec:em_DY}. 

\begin{figure}\centering
  \includegraphics[width=0.45\textwidth]{figures/ControlRegions/DY_check_mt}
  \includegraphics[width=0.45\textwidth]{figures/ControlRegions/ttbar_check_mt}
  \caption{\label{fig:mt_dy_tt} (Data driven W+Jets and QCD with only statistical uncertainties) 
Distributions of \meffmtau. Left:
    validation region with $\ETslash<30~\gev$, $n_b = 0$ and $60<\gev$ \meffmtau $<$ 150 GeV.  Right:
    validation region with $n_b\geq1$.}
\end{figure}

\subsubsection{Di-boson process}
Di-boson processes are a relatively small background in the muon + 
hadronic tau channel. They are estimated directly from simulation with 
a 20\% systematic uncertainty measured from the electron + muon channel 
in section \ref{section:diBoson}.
    
\subsection{QCD multi-jet background}\label{sec:mtau_qcd}
For a given variable and binning, e.g. the effective mass variable
used for signal extraction, we construct a data-driven template for
the shape of the QCD multi-jet background, i.e. the processes lacking
prompt leptons.  Based on the charge of the final state, the muon 
relative isolation and \tauh isolation, we split the events into 
four regions shown in Figure~\ref{fig:ABCDEF} and described below:
\begin{itemize}
  \item A (Signal) Region: $\mu$ and \tauh have opposite charge, $\mu$ relative isolation is less than $0.15$, and \tauh pass "Tight" isolation requirement.
  \item B Region: $\mu$ and \tauh have same charge, $\mu$ relative isolation is less than $0.15$, and \tauh pass "Tight" isolation requirement.
  \item E Region: $\mu$ and \tauh have opposite charge, $\mu$ relative isolation is between $0.15$ and $1.0$, and \tauh fail "Tight" isolation requirement.
  \item F Region: $\mu$ and \tauh have same charge, $\mu$ relative isolation is between $0.15$ and $1.0$, and \tauh fail "Tight" isolation requirement.
\end{itemize}

\begin{figure}\centering
  \includegraphics[width=0.9\textwidth]{figures/backgroundEstimation/ABCDEF}
  \caption{\label{fig:ABCDEF} Data driven QCD and W+jets estimation and validation strategy for the \teth, \tmth channels.}
\end{figure}
In region B, QCD events are estimated by subtracting events
with genuine leptons (estimated by simulation) bin-by-bin from data. In regions E and F, 
due to the very high purity of QCD events, QCD is estimated directly as the observed 
events. QCD events are assumed to be charge blind, thus, the amount of QCD events 
in region B should be comparable to that of in the signal region. However, with the 
freedom to define the anti-isolation region, we choose an anti-isolation definition 
such that QCD purity (98\%) is much higher compared to the signal region. 
Taking the QCD shape from region E will help us reduce the effects of potential 
mis-modeling of other backgrounds on QCD estimation.

Hence, QCD events in the signal region are estimated with the shape from region E 
(excluding MC contribution in region E as QCD purity is $> 98\%$) and multiplying 
a "QCD loose to tight" scale factor derived from regions B and F. Given the high 
purity (98\%) of QCD in region F, QCD is taken straight from data in region F. 
The factor is defined as:
\begin{equation}\label{eq:et_qcd_sf}
f_\mathrm{LT}^\mathrm{QCD} = \left(N_\mathrm{data}^\mathrm{B} - N_\mathrm{non-QCD~MC}^\mathrm{B}\right)
/ N_\mathrm{data}^\mathrm{F}\quad.
\end{equation}

As a summary, the QCD distribution in the signal region is estimated as the following:
\begin{equation}
(\text{QCD distribution})_{\text{A}} = f_\mathrm{LT}^\mathrm{QCD} \times (\text{data distribution})_{\text{E}} 
\end{equation}

Table \ref{table:SF_QCD_mt} shows the yields of data and MC samples in regions B 
and F used for the calculation of $f_\mathrm{LT}^\mathrm{QCD}$.

{\renewcommand{\arraystretch}{1.3}%
\begin{table}
   \caption{\label{table:SF_QCD_mt} Event yields in regions B and F used for the calculation of $f_\mathrm{LT}^\mathrm{QCD}$.}
   \centering{
     \begin{tabular}{ | l | c | c | }
        \hline \hline
        Process                         & region B          & region F      \\ \hline
        Z + jets                        & 189 $\pm$ 28      & 13 $\pm$ 8    \\ \hline
        $t\bar{t}$                      & 67 $\pm$ 5        & 16 $\pm$ 2    \\ \hline
        W + jets                        & 745 $\pm$ 134     & 76 $\pm$ 35   \\ \hline
        DiBoson                         & 19 $\pm$ 2        & 0.6 $\pm$ 0.3 \\ \hline
        non-QCD background              & 1020 $\pm$ 137    & 106 $\pm$ 36  \\ \hline
        Data                            & 2197              & 6452          \\ \hline \hline
        Data - non-QCD background       & 1177 $\pm$ 137    & 6452          \\ \hline
        $f_\mathrm{LT}^\mathrm{QCD}$    &  \multicolumn{2}{c|}{0.18 $\pm$ 0.02} \\ \hline
     \end{tabular}
   }
 \end{table}}

This QCD estimation method is valid only if the QCD shape in the 
anti-isolated region correctly models the QCD shape in the isolated
region. The check is done by comparing the observation and background
estimation in region B with the QCD shape taking from region F and
normalized to the QCD in region B.  An example of this test is shown
in the left panel of Fig.~\ref{fig:mt_dataDrivenChecks}. Here, W+jets 
is estimated directly from MC simulations. Overall, the data and 
estimated background agrees reasonably. Due to low statistics of 
the W+jets MC sample, some bins disagree by 20\%.

\begin{figure}\centering
  \includegraphics[width=0.45\textwidth]{figures/backgroundEstimation/mt/regionB_mt}
  \includegraphics[width=0.45\textwidth]{figures/backgroundEstimation/mt/WJets_SF_mt}
  \caption{\label{fig:mt_dataDrivenChecks} (Only statistical uncertainties are included) 
Distributions of \meffmtau. Left:
    in region B with signal region selections (data driven QCD and MC based W+jets).  Right:
    in region A' with signal region like selections (data driven QCD and W+jets).}
\end{figure}

\subsection{W+jets background}
\label{sec:mt_w_bkg_validation}

The simulated W+jets samples, especially at low \HT, were not
generated with large MC statistics.  If used directly, avoiding
non-smooth templates restricts somewhat the choice of signal selection
and binning.  It also complicates the validation of the background
estimates.  As a workaround, similar to the QCD estimation, we construct 
a data-driven template for the shape of the W+jets background in a control 
region with high W+jets purity. Based on the charge of the final state and 
\tauh isolation, we split the events into four regions shown in 
Fig.~\ref{fig:ABCDEF} and described below:
\begin{itemize}
  \item A (Signal) Region: $\mu$ and \tauh have opposite charge, $\mu$ relative isolation is less than $0.15$, and \tauh pass "Tight" isolation requirement.
  \item B Region: $\mu$ and \tauh have same charge, $\mu$ relative isolation is less than $0.15$, and \tauh pass "Tight" isolation requirement.
  \item C Region: $\mu$ and \tauh have opposite charge, $\mu$ relative isolation is less than $0.15$, and \tauh pass "VeryLoose" isolation requirement but fail "Tight" isolation requirement.
  \item D Region: $\mu$ and \tauh have same charge, $\mu$ relative isolation is less than $0.15$, and \tauh pass "VeryLoose" isolation requirement but fail "Tight" isolation requirement.
\end{itemize}

In most of the cases, W+jets passes our signal region selection by having a jet 
faking the reconstructed \tauh. Thus, by relaxing the \tauh ID, in region C and D, 
one would greatly increase the acceptance of W+jets events.

W+jets event in the signal region is estimated by applying a "jet to tau fake rate" 
to the estimated W+jets events in region C as the following:

\begin{equation}
(\text{W+jets distribution})_{\text{A}} = f_{\text{jet}\rightarrow\tau} \times (\text{W+jets distribution})_{\text{C}} 
\end{equation}

where the W+jets distribution in region C is estimated by subtracting 
non W+jets events with genuine leptons (estimated by simulation) and 
data-driven QCD events bin-by-bin from data. The data-driven QCD events 
in region C is estimated from the previously mentioned region E with a 
scale factor ($f^{\text{QCD}}_{\text{FtoD}}$) to properly transfer the 
yield from region E to C. The data-driven QCD estimation in region C 
can be summarized as: 

\begin{equation}
(\text{QCD distribution})_{\text{C}} = f^{\text{QCD}}_{\text{FtoD}} \times (\text{data distribution})_{\text{E}} 
\end{equation}
where $f^{\text{QCD}}_{\text{FtoD}} = \left(N_\mathrm{data}^\mathrm{D} - N_\mathrm{non-QCD~MC}^\mathrm{D}\right)/ N_\mathrm{data}^\mathrm{F}$

To properly estimate the most relevant (closest to our signal region as possible) 
"jet to tau fake rate" from data, we construct a W+jets rich region with the 
following selections:
\begin{itemize}
  \item $\cos{\Delta \phi (\mu,\tau_{h})}<-0.95$;
  \item $\ETslash>30~\gev$;
  \item $0.5 < \cos{\Delta \phi (\mu,\ETslash)} < 0.9~\text{and}~55~\gev~< m_{\text{T}}(\mu,\ETslash) < 120~\gev$;
  \item no jet with $p_T>30\gev$ tagged as a b-jet (CSV loose)
\end{itemize}
Events are further separated in to six regions (A', B', C', D', E' and F') by 
the same methods as shown in Fig.~\ref{fig:ABCDEF}. The "jet to tau fake rate" 
is estimated by the following:
\begin{equation} \label{eq:fakeRate}
f_{\text{jet}\rightarrow\tau} = \frac{N^{\text{A'}}_{\text{data}} - N^{\text{A'}}_{\text{non-W MC}} - N^{\text{E'}}_{\text{data}}\times f^{\text{QCD}}_{\text{LT}}}
{N^{\text{C'}}_{\text{data}} - N^{\text{C'}}_{\text{non-W MC}} - N^{\text{E'}}_{\text{data}}\times f^{\text{QCD}}_{\text{FtoD}}}
\end{equation}

Region A' is shown in the right panel of Fig.~\ref{fig:mt_dataDrivenChecks} 
with the "jet to tau fake rate" estimated as: $f_{\text{jet}\rightarrow\tau} = 0.25 \pm 0.02$.

\subsection{Validation of Background Estimations}
Additional to the validation tests shown in Fig.~\ref{fig:mt_dy_tt}, a test 
to simultaneously check the data driven QCD and W+jets estimations is performed 
by reverting the $\cos{\Delta \phi (\mu,\tauh)}$ cut in the following configuration:
\begin{itemize}
  \item $ -0.95 < \cos{\Delta \phi (\mu,\tauh)} < 0$
  \item $\ETslash>30~\gev$
  \item $\cos{\Delta \phi (\mu,\ETslash)} > 0.9~\text{or}~(\cos{\Delta \phi (\tauh,\ETslash)} > 0.9~\text{and}~m_{\text{T}}(\mu,\ETslash) > 150~\gev)$
  \item no jet with $p_T>30\gev$ tagged as a b-jet (CSV loose)
\end{itemize}

Figure~\ref{fig:revertCosDPhi_mt} shows the distributions of \meffmtau, \ETslash, 
$\mu$ \pt and  \tauh \pt in region A with the above set of selections. 

\begin{figure}\centering
  \includegraphics[width=0.45\textwidth,page=3]{figures/ControlRegions/revertCosDPhi_mt}
  \includegraphics[width=0.45\textwidth,page=4]{figures/ControlRegions/revertCosDPhi_mt}\\
  \includegraphics[width=0.45\textwidth,page=1]{figures/ControlRegions/revertCosDPhi_mt}
  \includegraphics[width=0.45\textwidth,page=2]{figures/ControlRegions/revertCosDPhi_mt}
  \caption{\label{fig:revertCosDPhi_mt} (Data driven W+Jets and QCD with only statistical uncertainties)
 Distributions with $ -0.95 < \cos{\Delta \phi (\mu,\tauh)} < 0$ selection. Top: left: 
\meffmtau right: \ETslash. Bottom: left: $\mu$ \pt right: \tauh \pt.}
\end{figure}

%%Figure~\ref{fig:signalRegion_mt} shows the distributions of \meffmtau in 
Figure~\ref{fig:SignalRegionPlot_a} shows the distributions of \meffmtau in 
the signal region.% with data blinded.

%\begin{figure}\centering
  %% \includegraphics[width=0.6\textwidth,page=1]{figures/signalRegion/mt_blind}
%  \includegraphics[width=0.6\textwidth,page=1]{figures/signalRegion/mt}
%  \caption{\label{fig:signalRegion_mt} (Data driven W+Jets and QCD with only statistical uncertainties)
% \meffmtau distribution with signal region selections.}
%\end{figure}
