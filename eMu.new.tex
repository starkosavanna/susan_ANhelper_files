\section{Electron + Muon Channel}\label{sec:eMu}
%\textbf{Responsible: E. Laird, Z. Mao}

\subsection{Event selection}\label{sec:em_selection}
The electron selection is identical to that described in
Section~\ref{sec:et_selection}.  Muons are required to have $p_{T}>
10$ GeV and $\vert \eta \vert < 2.1$, with a distance of closest
approach to the leading sum-$p_T^2$ primary vertex of less than
0.045~cm (transvese) and 0.2~cm (longitudinal), and to satisfy the
muon POG medium muon requirement. %, and to have relative isolation (including $\delta\beta$ corrections) $< 0.15$.

We build pairs of electrons and muons in which the electron and muon
are separated by at least $\Delta R > 0.3$.  In events with more than
one such pair, we select the pair with the two most isolated leptons,
considering first the muon, and then the electron.  This criterion was
seen to have good efficiency for signal samples.  In the rare case of
multiple such pairs having identical isolation values, the
reconstructed $p_T$'s are considred, preferring higher values.

After a pair has been chosen for an event, we require both the
electron and muon relative isolations to be $<0.15$, for an event to
enter the signal region.  To reduce a possible Drell-Yan background,
events are rejected if there is an additional electron satisfying the
requirements described in Section~\ref{sec:et_selection} and with
relative isolation $<0.3$, or an additional muon satisfying the above
identification requirements with relative isolation $<0.3$.

As for the other channels, the signal region is defined as having
\begin{itemize}
  \item $\cos{\Delta \phi (e,\mu)}<-0.95$
  \item $Q(e) \times Q(\mu) < 0$
  \item $\ETslash>30~\gev$
  \item $P_{\zeta}- 3.1 \times P_{\zeta}^{vis} > -50~\gev$
  \item no jet with $p_T>30\gev$ tagged as a b-jet (CSV loose)
\end{itemize}

\subsection{Genuine dilepton events}
Studies of simulated events indicate that for Drell-Yan process, top
quark single and pair production, and di-boson production, the
reconstructed and selected electrons and hadronic taus are typically
associated with genuine simulated leptons.  The nominal expected event
rates are estimated by scaling the simulated samples by the best
available cross sections, listed in Table~\ref{tab:mc_samples}, and by
the integrated luminosity of the data samples.

\subsubsection{Drell-Yan process}\label{sec:em_DY}
Systematics for Drell-Yan process is estimated in an Drell-Yan rich 
region with the following selections and shown in the left panel of 
Figure~\ref{fig:em_dy_tt}:
\begin{itemize}
  \item $Q(e) \times Q(\mu) < 0$
  \item $\ETslash<30~\gev$
  \item no jet with $p_T>30\gev$ tagged as a b-jet (CSV loose)
  \item \meffemu $<$ 125 GeV
\end{itemize}
The Drell-Yan production rate systematics estimated to be:
\begin{equation}\label{eq:DY}
\text{Drell-Yan systematics} = \left| 1 - \frac{\text{Drell-Yan}}{\text{Data - other backgrounds}}\right| = 12\%
\end{equation}
which we apply both to \tetm and \teth final states.

\subsubsection{$t\bar{t}$ and single top processes}\label{sec:em_tt}
Systematics for $t\bar{t}$ and single top processes are estimated in an top quark rich 
region with the following selections and shown in the right panel of 
Figure~\ref{fig:em_dy_tt}:
\begin{itemize}
  \item $Q(e) \times Q(\mu) < 0$
  \item $\ETslash>30~\gev$
  \item $P_{\zeta}- 3.1 \times P_{\zeta}^{vis} < -50~\gev$
  \item at least one jet with $p_T>30\gev$ tagged as a b-jet (CSV loose)
\end{itemize}
The $t\bar{t}$ + single top production rate systematics estimated to be:
\begin{equation}\label{eq:tt}
\text{$t\bar{t}$ + single top systematics} = \left| 1 - \frac{\text{$t\bar{t}$ + single top}}{\text{Data - other backgrounds}}\right| = 8\%
\end{equation}
which we apply both to \tetm and \teth final states.

\subsubsection{Di-boson process}
We take di-boson processes directly from simulation with a 15\% production uncertainty.


\begin{figure}\centering
  \includegraphics[width=0.45\textwidth]{figures/et-em/Data_MC_Comparison/em_DY_Validation_Met_lessThan30_nCSVL_lessThan1_closeUp}
  \includegraphics[width=0.45\textwidth]{figures/et-em/Data_MC_Comparison/em_TT_Validation}
  \caption{\label{fig:em_dy_tt} Distributions of \meffemu. Left:
    validation region with $\ETslash<30~\gev$, $n_b = 0$ and \meffemu $<$ 125 GeV.  Right:
    validation region with $\ETslash>30~\gev$, $n_b\geq1$ and $P_{\zeta}- 3.1 \times P_{\zeta}^{vis} < -50~\gev$.}
\end{figure}




\subsection{QCD multi-jet background}\label{sec:em_qcd}
The estimation of the QCD background for the \tetm channel is directly
analogous to that in the \teth channel, except that the sideband is
defined by the muon isolation.  Figure~\ref{fig:em_scans} shows the
results of the scanning for a sideband.  The range of relative
isolation from 0.15 to 0.95 was chosen as the sideband. After the signal 
region selection the "Loose-to-Tight" scale factor is estimated
to be: $0.20 \pm 0.08$ where an additional 40\% uncertainty is added to the QCD systematics 
on top of the bin-by-bin systematics.

\begin{figure}\centering
  \includegraphics[width=0.45\textwidth]{figures/et-em/antiIsolationScan/em_SS_chi2Scan_SF}
  \includegraphics[width=0.45\textwidth]{figures/et-em/antiIsolationScan/em_SS_chi2Scan_p_value}
  \caption{\label{fig:em_scans} \tetm channel: scan of the range of
    relaxed relative isolation for the muon.  Left: text within each
    bin gives the normalization factor applied to same-charge
    iso-relaxed events (``loose to tight'' factor); the color axis
    matches the right plot.  Right: $\chi^2$ of the agreement between
    the predicted and observed distributions of tightly-isolated
    same-charge events.}
\end{figure}


\subsection{W+jets and validation of total estimated background}
\label{sec:em_w_bkg_validation}
In contrast to the \teth channel, the W background is very small, and
is estimated directly from the simulation.

The expected SM event yields in the signal region, in a fixed-width
binning, are shown in Figure~\ref{fig:em_sm_template}.

\begin{figure}\centering
  \includegraphics[width=0.45\textwidth]{figures/et-em/backgroundTemplates/em_puweight_OS_signalRegion_singleEle}
  \caption{\label{fig:em_sm_template} Expected event yields for the SM
    processes in the \tetm channel.  The data histogram is not shown
    in order to remain ``blind.''}
\end{figure}

