\section{Electron + Muon Channel}\label{sec:eMu}
%\textbf{Responsible: E. Laird, Z. Mao, U. Heintz}

\subsection{Event selection}\label{sec:em_selection}

The electron selection is identical to that described in
Section~\ref{sec:et_selection}.  Muons are required to have:
\begin{itemize}
  \item $p_{T} > 10$ GeV and $\vert \eta \vert < 2.1$
  \item distance of closest approach to the leading sum-$p_T^2$ 
    primary vertex of less than 0.045~cm (transvese) and 0.2~cm (longitudinal)
  \item satisfy the muon POG medium muon requirement
\end{itemize}

We build pairs of electrons and muons in which the electron and muon
are separated by at least $\Delta R > 0.5$.  In events with more than
one such pair, we select the pair with the two most isolated leptons,
considering first the muon, and then the electron.  This criterion was
seen to have good efficiency for signal samples.  In the rare case of
multiple such pairs having identical isolation values, the
reconstructed $p_T$'s are considred, preferring higher values.

After a pair has been chosen for an event, we require both the
electron and muon relative isolations to be $<0.15$, for an event to
enter the signal region.  To reduce a possible Drell-Yan background,
events are rejected if there is an additional electron satisfying the
requirements described in Section~\ref{sec:et_selection} and with
relative isolation $<0.3$, or an additional muon satisfying the above
identification requirements with relative isolation $<0.3$. To further 
reduce backgrounds, we require the electron and muon to have 
opposite charge. The selection set mentioned above is defined as preselection.

Following the preselection, as for the \thth channel, the signal 
region is defined as having:
\begin{itemize}
  \item $\cos{\Delta \phi (e,\mu)}<-0.95$;
  \item $\ETslash>30~\gev$;
  \item $\cos{\Delta \phi (\pt\text{ leading lepton},\ETslash)}<-0.9$;
  \item no jet with $p_T>30\gev$ tagged as a b-jet (CSV loose)\quad.
\end{itemize}

The distributions of these variables after preselection, and after
selection requirements on the other variables, are shown in
Figure~\ref{fig:em_nm1_distributions}.

\begin{figure}\centering
  \includegraphics[width=0.31\textwidth]{figures/n_1/em/met_em}
  \includegraphics[width=0.31\textwidth]{figures/n_1/em/cosDPhi_em}
  \includegraphics[width=0.31\textwidth]{figures/n_1/em/nb_em}
  \caption{\label{fig:em_nm1_distributions} (Data driven QCD with only statistical uncertainties) Distributions of the
    variables used for \teth signal selection, after all other signal
    selection requirements on variables other than the one plotted:
    \ETslash (left), $\cos{\Delta \phi (e,\mu)}$ (middle, with $\cos{\Delta \phi (e, \mu)} > 0$) and $n_b$ (right).}
\end{figure}

\subsection{Genuine dilepton events}
Studies of simulated events indicate that for Drell-Yan process, top
quark single and pair production, and di-boson production, the
reconstructed and selected muons and electrons are typically
associated with genuine simulated leptons.  The nominal expected event
rates are estimated by scaling the simulated samples by the best
available cross sections, listed in Table~\ref{tab:mc_samples}, and by
the integrated luminosity of the data samples.

\subsubsection{Drell-Yan process}\label{sec:em_DY}
Systematics for Drell-Yan process is estimated in an Drell-Yan rich 
region with the following selections and shown in the left panel of 
Figure~\ref{fig:em_dy_tt}:
\begin{itemize}
  \item $Q(e) \times Q(\mu) < 0$;
  \item $\ETslash<30~\gev$;
  \item no jet with $p_T>30\gev$ tagged as a b-jet (CSV loose);
  \item $60~\gev <$ \meffemu $<150~\gev$\quad.
\end{itemize}

\begin{figure}\centering
  \includegraphics[width=0.45\textwidth]{figures/ControlRegions/DY_check_em}
  \includegraphics[width=0.45\textwidth]{figures/ControlRegions/ttbar_check_em}
  \caption{\label{fig:em_dy_tt} (Data driven QCD with only statistical uncertainties) Distributions of \meffetau. Left:
    validation region with $\ETslash<30~\gev$, $n_b = 0$ and $60<$ \gev \meffemu $<$ 150 GeV.  Right:
    validation region with $n_b\geq1$.}
\end{figure}

The Drell-Yan production rate difference between data and MC is estimated to
be:
\begin{equation}\label{eq:DY}
\frac{\text{Data - non-Drell-Yan backgrounds}}{\text{MC Drell-Yan}} = 0.98 \pm 0.05\%\quad.
\end{equation}
Thus, taking the largest two, whether it's the deviation from nominal or the 
uncertainty, the Drell-Yan production rate systematic uncertainty is estimated 
to be 5\%, which we apply to all final states.

\subsubsection{$t\bar{t}$ and single top processes}\label{sec:em_tt}
Systematics for $t\bar{t}$ and single top processes are estimated in a
top quark rich region with the following selections and shown in the
right panel of Figure~\ref{fig:em_dy_tt}:
\begin{itemize}
  \item $Q(e) \times Q(\mu) < 0$;
  \item $\ETslash>30~\gev$;
  \item $\cos{\Delta \phi (\pt\text{ leading lepton},\ETslash)}<-0.9$;
  \item at least one jet with $p_T > 30\gev$ tagged as a b-jet (CSV loose).
\end{itemize}
As described in ~\ref{sec:mt_tt}, top pt reweighting is applied to simulated 
\ttbar events. Good agreement between data and background predictions is 
observed.
The $t\bar{t}$ + single top data/MC overall agreement is estimated to be:
\begin{equation}\label{eq:em_tt}
\frac{\text{Data - other backgrounds}}{\text{$t\bar{t}$ + single top}} = 1.00 \pm 0.01 \quad.
\end{equation}
Thus, taking the largest two, whether it's the deviation from nominal or the 
uncertainty, the $t\bar{t}$ + single top production rate systematic uncertainty 
is estimated to be 1\%, which we apply to all final states.

\subsubsection{Di-boson process}\label{section:diBoson}
Di-boson processes are estimated using simulated MC samples. In order to 
quantify the agreement between simulated and observed di-boson events, a 
di-boson rich region is constructed with the following selections: 
\begin{itemize}\label{table:diboson}
  \item $Q(e) \times Q(\mu) < 0$;
  \item $\ETslash>30~\gev$;
  \item no jet with $p_T > 20\gev$;
  \item $P_{\zeta}- 3.1 \times P_{\zeta}^{vis} < -50$;
  \item $\cos{\Delta \phi (\pt\text{ leading lepton},\ETslash)} > -0.9$;
  \item \meffemu $>$ 200 GeV.
\end{itemize}

Additional to the usual electric charge and \ETslash requirements, events containing 
jets with $p_T > 20\gev$ and $|\eta| < 2.1$ are vetoed to reduce $t\bar{t}$ 
contamination. This serves as a tighter $t\bar{t}$ reduction as oppose to the 
b-jet vetos. The motivation of $P_{\zeta}- 3.1 \times P_{\zeta}^{vis} < -50$ 
can be seen in the left panel of Fig \ref{fig:diBoson} which shows the 
distribution of $P_{\zeta}- 3.1 \times P_{\zeta}^{vis}$. By cutting at -50, 
we remove almost all the signal contamination while keeping ~50\% of the 
di-boson events. The $\cos{\Delta \phi (\pt\text{ leading lepton},\ETslash)} > -0.9$ 
cut ensures this control region is orthogonal to the signal region. Finally, 
the \meffemu $>$ 200 GeV cut is chosen to reject Drell-Yan, QCD and W+jets events 
and improve di-boson purity (63\%).

\begin{figure}\centering
  \includegraphics[width=0.45\textwidth]{figures/backgroundEstimation/em/zeta}
  \includegraphics[width=0.45\textwidth]{figures/backgroundEstimation/em/diBoson}
  \caption{\label{fig:diBoson} (Data driven QCD with only statistical uncertainties) 
Left: $P_{\zeta}- 3.1 \times P_{\zeta}^{vis}$ distribution after requiring $Q(e) \times Q(\mu) < 0$, 
$\ETslash>30~\gev$ and no jet with $p_T > 20\gev$. 
Right: \meffemu distribution after all the requirements mentioned in \ref{table:diboson}.}
\end{figure}

The right panel of Fig \ref{fig:diBoson} shows the \meffemu distribution after all 
the requirements mentioned in \ref{table:diboson}. Overall, we see a reasonable agreement 
between data and the expected backgrounds. The ratio between data and MC for di-boson 
processes is estimated to be:

\begin{equation}\label{eq:em_diboson}
\frac{\text{Data - other backgrounds}}{\text{di-boson MC}} = 1.01 \pm 0.20 \quad.
\end{equation}

Taking the shape variation into consideration, we estimate a 20\% systematic uncertainty 
for di-boson processes.

To cross check the di-boson study, a WZ enrich region is constructed 
with the following requirements:
\begin{itemize}\label{table:diboson2}
  \item $Q(e) \times Q(\mu) > 0$;
  \item $\ETslash>30~\gev$;
  \item at least three leptons with at least one electron and one muon;
  \item no jet with $p_T>30\gev$ tagged as a b-jet (CSV loose);
  \item \meffemu $>$ 200 GeV.
\end{itemize}

\begin{figure}\centering
  \includegraphics[width=0.6\textwidth]{figures/backgroundEstimation/em/diboson_WZ}
  \caption{\label{fig:diBoson2} (Data driven QCD with only statistical uncertainties) 
\meffemu distribution after all the requirements mentioned in \ref{table:diboson2}.}
\end{figure}

By requesting at least three leptons and of the two selected carries the 
same electric charge, most of Drell-Yan, W+jets and $t\bar{t}$ backgrounds 
are rejected with a di-boson purity of 84\%. Fig \ref{fig:diBoson2} shows 
the \meffemu distribution after all the requirements mentioned in 
\ref{table:diboson2} where we see a good agreement between data and 
the expected backgrounds. The ratio between data and MC for di-boson 
processes in the WZ enriched region is estimated to be:

\begin{equation}
\frac{\text{Data - other backgrounds}}{\text{di-boson MC}} = 0.98 \pm 0.04 \quad.
\end{equation}
which agrees well with the results in equation \ref{eq:em_diboson}.

\subsection{QCD multi-jet background}\label{sec:em_qcd}
The estimation of the QCD background for the \tetm channel is directly
analogous to that in the \teth and \tmth channels, except that the 
sideband is defined by the muon isolation.  As shown in the left panel 
of Fig. \ref{fig:em_bkgEstimation}, in order to achieve good QCD purity, 
the relative isolation sideband of 0.15 to 1.0 was chosen. After the
signal region selection the "Loose-to-Tight" scale factor is estimated
to be: $0.23 \pm 0.05$ where this 23\% rate uncertainty is applied to
the QCD process (in addition to the bin-by-bin systematic
uncertainties).

Table \ref{table:SF_QCD_em} shows the yields of data and MC samples in 
isolated (muon iso $<$ 0.15) and anti-isolated (0.15 $<$ muon iso $<$ 1.0) 
regions used for the calculation of the "Loose-to-Tight" scale factor.

{\renewcommand{\arraystretch}{1.3}%
\begin{table}
   \caption{\label{table:SF_QCD_em} Event yields in isolated and anti-isolated regions used for the calculation of the "Loose-to-Tight" scale factor.}
   \centering{
     \begin{tabular}{ | l | c | c | }
        \hline \hline
        Process                         & muon relative iso $<$ 0.15  & 0.15 $<$ muon relative iso $<$ 1.0 \\ \hline
        Z + jets                        & 27 $\pm$ 10               & 49 $\pm$ 14       \\ \hline
        $t\bar{t}$                      & 26 $\pm$ 3                & 241 $\pm$ 8       \\ \hline
        W + jets                        & 52 $\pm$ 39               & 209 $\pm$ 75      \\ \hline
        DiBoson                         & 41 $\pm$ 3                & 6 $\pm$ 1         \\ \hline
        non-QCD background              & 146 $\pm$ 40              & 505 $\pm$ 77      \\ \hline
        Data                            & 370                       & 1490              \\ \hline \hline
        Data - non-QCD background       & 224 $\pm$ 40              & 985.0 $\pm$ 77    \\ \hline
        "Loose-to-Tight" scale factor   &  \multicolumn{2}{c|}{0.23 $\pm$ 0.05}           \\ \hline
     \end{tabular}
   }
 \end{table}}


\begin{figure}\centering
  \includegraphics[width=0.45\textwidth]{figures/backgroundEstimation/em/muonIso}
  \includegraphics[width=0.45\textwidth]{figures/backgroundEstimation/em/WJets_MC}
  \caption{\label{fig:em_bkgEstimation} (Data driven QCD with only statistical uncertainties) 
Left: muon relative isolation with QCD estimated from same-sign region. 
Right: W+jets MC closure test between isolated and anti-isolated events.}
\end{figure}

\subsection{W+jets background}
\label{sec:em_w_bkg_validation}
The W background is very small.  However, the W+jets simulated sample was not 
generated with high statistics.  As a workaround, the W+jets shape is taken 
from the simulated sample in the muon isolation sideband (muon relative 
isolation from 0.15 to 1.0), and scaled to match the simulated yield in the 
tight muon isolation (muon relative isolation $<$ 0.15).  The ``loose-to-tight'' 
factor is $0.21 \pm 0.07$. Thus, we improve the statistics of the W+jets 
template by $1/0.21 = 4.8$ times. A shape comparison of the \meffemu distribution 
between the isolated and anti-isolated W+jets MC events is shown in the right 
panel of Fig. \ref{fig:em_bkgEstimation}. As shown, the W+jets shape from the 
isolation sideband does a reasonable job at modeling the shape of the isolated 
region.

\subsection{Validation of Background Estimations}
Additional to the validation tests shown in Fig.~\ref{fig:em_dy_tt}, a test 
to check the overall performance of the background estimations is constructed 
by reverting the $\cos{\Delta \phi (e,\mu)}$ cut as the following:
\begin{itemize}
  \item $ -0.95 < \cos{\Delta \phi (e,\mu)} < 0$
  \item $\ETslash>30~\gev$;
  \item $\cos{\Delta \phi (\pt\text{ leading lepton},\ETslash)}<-0.9$;
  \item no jet with $p_T>30\gev$ tagged as a b-jet (CSV loose)\quad.
\end{itemize}

Figure~\ref{fig:revertCosDPhi_em} shows the distributions of \meffemu, \ETslash, 
electron \pt and muon \pt in the signal like region with the above set of selections. 

\begin{figure}\centering
  \includegraphics[width=0.45\textwidth,page=3]{figures/ControlRegions/revertCosDPhi_em}
  \includegraphics[width=0.45\textwidth,page=4]{figures/ControlRegions/revertCosDPhi_em}\\
  \includegraphics[width=0.45\textwidth,page=1]{figures/ControlRegions/revertCosDPhi_em}
  \includegraphics[width=0.45\textwidth,page=2]{figures/ControlRegions/revertCosDPhi_em}
  \caption{\label{fig:revertCosDPhi_em} (Data driven QCD with only statistical uncertainties) 
 Distributions with $-0.95 < \cos{\Delta \phi (e,\mu)} < 0$ selection. Top: left: 
\meffemu right: \ETslash. Bottom: left: e \pt right: $\mu$ \pt.}
\end{figure}

%%Figure~\ref{fig:signalRegion_em} shows the distributions of \meffmtau in 
Figure~\ref{fig:SignalRegionPlot_a} shows the distributions of \meffmtau in 
the signal region.% with data blinded.

%\begin{figure}\centering
  %% \includegraphics[width=0.6\textwidth,page=1]{figures/signalRegion/em_blind}
%  \includegraphics[width=0.6\textwidth,page=1]{figures/signalRegion/em}
%  \caption{\label{fig:signalRegion_em} (Data driven QCD with only statistical uncertainties)
% \meffemu distribution with signal region selections.}
%\end{figure}


\iffalse
\subsection{Overlays of observations and SM predictions}
\label{sec:em_overlays}

The expected SM event yields in the signal region are shown in
Figure~\ref{fig:em_sm_template_and_mt}.
\begin{figure}\centering
  \includegraphics[width=0.45\textwidth]{figures/bkgTemplate_ZPrime1500_em} \\
  \includegraphics[width=0.45\textwidth]{figures/et-em/kinematicDistributions_unblind/em_puweight_OS_signalRegion_mt_e_met}
  \includegraphics[width=0.45\textwidth]{figures/et-em/kinematicDistributions_unblind/em_puweight_OS_signalRegion_mt_mu_met}
  \caption{\label{fig:em_sm_template_and_mt} Top: predicted background
    yields and observed event yields in the \tetm channel after signal
    selection.  Bottom: distributions of transverse mass.}
\end{figure}

Distributions of \pt and $\eta$ are shown in Figure~\ref{fig:em_sr_pt_eta}.
\begin{figure}\centering
  \includegraphics[width=0.45\textwidth]{figures/et-em/kinematicDistributions_unblind/em_puweight_OS_signalRegion_ePt}
  \includegraphics[width=0.45\textwidth]{figures/et-em/kinematicDistributions_unblind/em_puweight_OS_signalRegion_eEta} \\
  \includegraphics[width=0.45\textwidth]{figures/et-em/kinematicDistributions_unblind/em_puweight_OS_signalRegion_mPt}
  \includegraphics[width=0.45\textwidth]{figures/et-em/kinematicDistributions_unblind/em_puweight_OS_signalRegion_mEta}
  \caption{\label{fig:em_sr_pt_eta} Distributions, after \tetm final
    selection, of electron \pt (top left), electron pseudo-rapidity
    (top right), muon \pt (bottom left), muon pseudo-rapidity (bottom
    right).}
\end{figure}
\fi
