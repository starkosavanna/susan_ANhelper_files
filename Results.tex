\section{Results and Conclusions}\label{sec:results}
%\textbf{Responsible: T. Kamon, M. Dalchenko}

Figure\,\ref{fig:SignalRegionPlot_a} shows the background predictions as well as the observed $m(\tau_{1},\tau_{2},\MET)$ spectrum, in log scale, for the four 
channels considered in this analysis note: \mutau (top left), \ditauhad (top right), \etau (bottom left), \emu (bottom right). 


\iffalse

Figure\,\ref{fig:SignalRegionPlot_a} shows the background predictions as well as the observed $m(\tau_{1},\tau_{2},\MET)$ spectrum, in log scale, for the four 
channels considered in this analysis note: \mutau (top left), \ditauhad (top right), \etau (bottom left), \emu (bottom right). 
Table ~\ref{table:EvtSR} (top) lists the number of estimated background events compared with the total number of observed events in data for each final state
considering the whole mass spectrum, while Table ~\ref{table:EvtSR} (bottom) those considering $M(\tau_1,\tau_2,\not\!\!E_T) >$ 300 \GeV. 
Table ~\ref{tab:summaryTable} lists the predicted signal yields for various $Z^{\prime}$ masses. 
%Figure\,\ref{fig:SignalRegionPlot_b} shows the same unblinded result, except in normal scale in order to emphasize the low mass region. 
%(\textbf{placeholder text:} we are still blinded). 
The observed $m(\tau_{1},\tau_{2},\MET)$ spectrum in the signal region does not reveal any evidence for $Z^{\prime}\to\tau\tau$ production 
%(\textbf{this sentence and what follows is just placeholder text since we have not unblinded}). 
%The total predicted background rate is $19.8 \pm 4.2$, with QCD multijet,
%$t\bar{t}$, and $Z\to\tau\tau$ composing $\sim$ 76.3\%, 6.6\%, and 12.6\% of the rate respectively (see Table~\ref{table:expectations}).
%The distribution from the $m(W_{R})=2700$ GeV signal hypothesis is plotted with the background prediction in Figure\,\ref{fig:SignalRegionPlot} to illustrate how
%a hypothetical signal would appear above the SM background prediction. 
\begin{table}[hc]
\topcaption{Number of observed events in data and estimated background events, for the whole mass range (top) and
in the region $M(\tau_{1},\tau_{2},\not\!\!E_T) >$ 300 GeV (bottom).
The uncertainties quoted on the number of background events represent the combined statistical and systematic uncertainty.
}
\begin{center}
\begin{tabular}{cccccc}
\hline
Process    & $\tau_h \tau_h$ & \mutau & \etau & \emu & \\
\hline
Drell-Yan  & 8    $\pm$ 3    & 882    $\pm$ 127   & 375    $\pm$ 40     & 321   $\pm$ 37    & \\
%W+jets     & 0.1  $\pm$ 0.1  & 916    $\pm$ 96    & 520    $\pm$ 31     & 19    $\pm$ 6     & \\
W+jets     & 0.1  $\pm$ 0.1  & 916    $\pm$ 96    & 456    $\pm$ 35     & 19    $\pm$ 6     & \\
Diboson    & 0.5  $\pm$ 0.5  & 29     $\pm$ 7     & 18     $\pm$ 4      & 108   $\pm$ 11    & \\
$t\bar{t}$ & --              & 26     $\pm$ 7     & 26     $\pm$ 6      & 223   $\pm$ 20    & \\
%Multijet   & 49   $\pm$ 13   & 122    $\pm$ 84    & 199    $\pm$ 36     & 36    $\pm$ 16     & \\
Multijet   & 49   $\pm$ 13   & 122    $\pm$ 84    & 250    $\pm$ 50     & 36    $\pm$ 16     & \\
\hline
Total      & 58   $\pm$ 13   & 1976   $\pm$ 180   & 1125   $\pm$ 73     & 707   $\pm$ 47     & \\
%Total      & 58   $\pm$ 13   & 1976   $\pm$ 180   & 1138   $\pm$ 63     & 707   $\pm$ 47     & \\
\hline
Observed   & 55              & 1807               & 1113                & 728               & \\
\hline
 & & & & & \\
\hline
Process    & $\tau_h \tau_h$ & \mutau & \etau & \emu & \\
\hline
Drell-Yan  & 5     $\pm$ 2      & 16 $\pm$ 4    & 9  $\pm$ 4    & 4    $\pm$ 3      & \\
W+jets     & 0.004 $\pm$ 0.004  & 23 $\pm$ 9    & 7  $\pm$ 5    & 0.2  $\pm$ 0.5  & \\
Diboson    & 0.02  $\pm$ 0.02   & 6  $\pm$ 3    & 3  $\pm$ 2    & 23   $\pm$ 5      & \\
$t\bar{t}$ & --                 & 4  $\pm$ 2    & 5  $\pm$ 3    & 65   $\pm$ 12 & \\
Multijet   & 18    $\pm$ 6      & 4  $\pm$ 3    & 9  $\pm$ 3    & 0.8  $\pm$ 1.0  & \\
\hline
Total      & 23    $\pm$ 6      & 51 $\pm$ 11   & 33 $\pm$ 8    & 93 $\pm$ 13 & \\
\hline
Observed   & 20                 & 42            & 40            & 96              & \\
\hline
\end{tabular}

\end{center}
\label{table:EvtSR}
\end{table}

\fi



An upper bound at 95\% confidence level (CL) is set on $\sigma \cdot BR$, 
where $\sigma$ is the cross-section for pair production of $pp\to Z^{\prime}$ and $BR$ the branching fraction for $Z^{\prime}\to\tau\tau$. We use a binned maximum 
likelihood fit for the signal plus background and background only hypotheses. We follow a modified frequentist approach, known as the CL$_{s}$ method. The 
likelihood is the product of Poisson probabilities 
%$P(n_{i}|\nu_{i}(\mu,\theta))$ 
$P(n_{i}|\nu_{i}(\mu,\theta)) = \frac{\nu_{i}^{n_{i}}}{n_{i}!}$exp$(-\nu_{i})$
to observe $n_{i}$ events in each bin $i$ of the mass distribution and where we denote $\nu_{i}$ as the yields expected in that bin (sum of expected signal plus 
expected background).

The calculation of the exclusion limit is obtained by using each bin of the $m(\tau_{1},\tau_{1},\MET)$ distribution to construct one bin of the likelihood 
and computing the 95\% confidence level (CL) upper limit on the signal cross-section using the asymptotic CL$_{s}$ method. 
%in the official limit calculation tool ``combine". 
Therefore, a shape based analysis is performed, using the $m(\tau_{1},\tau_{1},\MET)$ distribution as the fit discriminant to determine the 
likelihood of observing signal in the presence of the predicted background rate, given the observed yield in data. Systematic uncertainties are represented by nuisance parameters, which are profiled, assuming a log normal prior for normalization parameters, and Gaussian priors for mass-spectrum shape uncertainties.

The above procedure is performed using the Higgs limit calculation tool ``combine". The tool takes as input data cards with the total yields and $m(\tau_{1},\tau_{1},\MET)$ distributions (signal, 
background, and data) in each channel, along with the nuissance parameters for systematic uncertainties. 
%and nuisance parameters in 
%each $m(\tau_{1},\tau_{1},\MET)$ bin of the search channels. 

These data cards are provided by the four channels being considered. 
The individual limits, per channel, were obtained by running the combine tool over each of these cards separately. 
%As only an example, if all the channels had 10 bins from $0 < m(\tau_{1},\tau_{1},\MET) < 5000$ GeV 
%(500 GeV/bin), this means there will be 10 cards per channel and therefore 40 cards in total for the four channels (if all the channels have the same bin size).
The cards for the four channels are then combined using the ``CombineCards.py" tool provided by the Higgs limit tool, 
resulting in a single \textit{combined} data card. 
The only further input that is required are the correlations within and across channels. 
%This procedure was performed for all the final states considered in the analysis, resulting in 
%X \textit{combined} cards. 
%The individual limits, per channel, were obtained by running the combine tool over each \textit{combined} card separately. The final 
%combined limit was obtained by combining the four resulting \textit{combined} cards per channel described above, using the ``CombineCards.py" tool. In order to 
%handle correlations within and across channels, the following approach was used. Each nuisance parameter was defined with a convention of two indices: the first 
%index, $i$, denoted the channel ($i=\mu\tau_{h}=0$, $i=\tau_{h}\tau_{h}=1$, $i=$e$\tau_{h}=2$, $i=$e$\mu=3$) and the second one the type of process 
%($j=$Signal$=0$, $j=$W+jets$=1$, etc.). Since the limit tool handles nuisance parameters with the same name as fully correlated, correlations across channels and 
%processes were specified by utilizing the same $i$ and process index $j$, respectively.

Figure\,\ref{fig:Limits} shows the expected limits as well as the leading order theoretical cross-section as functions of $m(Z^{\prime})$ mass for each channel. 
\textbf{Since the analysis is currently blinded}, the ``observed" limits in these plots are merely pseudo-data using a background-only hypothesis. 
%Figure\,\ref{fig:CombinedLimits} show the combined limit. A k factor of 1.3 has been used to scale the leading order (LO) signal cross-section. 
We expect to exclude $Z^{\prime}_{\textrm{SSM}}$ (decaying through $Z^{\prime}\to\tau\tau$) masses below approximately 2.7, 2.65, 2.6, and 1.75 TeV with the 
\ditauhad, $\tau_{\mu} \tauh$, $\tau_\Pe \tauh$, and $\tau_\Pe \tau_{\mu}$ channels respectively.

%\iffalse
\begin{figure}[tbh!]
  \centering
   \includegraphics[width=0.4\textwidth]{figures/signalRegion/mt}
%   \includegraphics[width=0.4\textwidth]{figures/backgroundEstimation/tautau/SR_unblinded_log.pdf} 
   \includegraphics[width=0.4\textwidth]{figures/signalRegion/tt} 
   \includegraphics[width=0.4\textwidth]{figures/signalRegion/et}
   \includegraphics[width=0.4\textwidth]{figures/signalRegion/em}
  \caption{ Top Left: $m(\mu,\tau_{h},\MET)$ distribution in the signal region, in log scale.  Top Right: $m(\tau_{h},\tau_{h},\MET)$ distribution in 
the signal region, in log scale.  Bottom Left: $m(e,\tau_{h},\MET)$ distribution in the signal region, in log scale.  Bottom Right: $m(e,\mu,\MET)$ 
distribution in the signal region, in log scale.}
    \label{fig:SignalRegionPlot_a}
\end{figure}

\iffalse
\begin{table}[ht]
\begin{center}
  \caption{Summary table of predicted signal yields after final selections. The signal acceptance is taken from LO samples, while a NLO k-factor of 1.3 has been used for the normalization.\label{tab:summaryTable}}
  \begin{tabular}{| l | c | c | c | c |}
  \hline
       Process          & \thth             & \tmth                 & \teth                 & \tetm            \\ \hline
       Z' (500)         & 399.6 $\pm$ 45.9  & 653.0 $\pm$ 75.0      & 256.9 $\pm$ 29.5      & 284.2 $\pm$ 35.5  \\   
       Z' (1000)        & 45.0 $\pm$ 3.4    & 53.0 $\pm$ 4.0        & 19.1 $\pm$ 1.4        & 24.7 $\pm$ 1.9        \\   
       Z' (1500)        & 8.6 $\pm$ 0.4     & 9.4 $\pm$ 0.4         & 3.0 $\pm$ 0.1         & 4.7 $\pm$ 0.3      \\   
       Z' (2000)        & 2.1 $\pm$ 0.09    & 2.3 $\pm$ 0.1         & 0.77 $\pm$ 0.04       & 1.2 $\pm$ 0.05        \\   
       Z' (2500)        & 0.72 $\pm$ 0.03   & 0.78 $\pm$ 0.03       & 0.25 $\pm$ 0.01       & 0.39 $\pm$ 0.01       \\   
       Z' (3000)        & 0.17 $\pm$ 0.01   & 0.18 $\pm$ 0.01       & 0.05 $\pm$ 0.002      & 0.09 $\pm$ 0.003   \\   
%       Drell-Yan        & 8.4 $\pm$ 3.1     & 882.4 $\pm$ 127.0     & 375.1 $\pm$ 117.6     & 321.2 $\pm$ 99.2 \\
%       W+jets           & 0.1 $\pm$ 0.1     & 916.2 $\pm$ 96.1      & 545.8 $\pm$ 85.6      & 18.9 $\pm$ 11.4 \\
%       Diboson          & 0.5 $\pm$ 0.5     & 29.2 $\pm$ 7.4        & 18.0 $\pm$ 4.4        & 108.3 $\pm$ 17.4 \\
%       $t\bar{t}$       & --                & 26.1 $\pm$ 6.7        & 26.1 $\pm$ 7.5        & 222.8 $\pm$ 44.8 \\
%       Multijet         & 48.7 $\pm$ 13.0   & 121.8 $\pm$ 83.5      & 116.7 $\pm$ 71.5      & 31.9 $\pm$ 24.3 \\
%       Observation      & 55                & 1807                  & 1113                  & 728        \\
  \hline
  \end{tabular}
\end{center}
\end{table}
\fi

\begin{figure}[tbh!]
  \centering
  \includegraphics[width=0.4\textwidth]{figures/Limits/mt.pdf}
  \includegraphics[width=0.4\textwidth]{figures/Limits/tt}\\
  \includegraphics[width=0.4\textwidth]{figures/Limits/et.pdf}
  \includegraphics[width=0.4\textwidth]{figures/Limits/em.pdf}
  \caption{Observed and expected limits for the $\tau_{\mu}\tau_{h}$,
    $\tau_{h}\tau_{h}$, $\tau_{e}\tau_{h}$, $\tau_{e}\tau_{\mu}$
    channels.  %A k-factor of 1.3 has been used to scale the leading order (LO) signal cross-section.
    }
    \label{fig:Limits}
\end{figure}

\iffalse
%\begin{figure}[tbh!]
%  \centering
%  \includegraphics[width=0.6\textwidth]{figures/limits/CombinedLimit}  \caption{Combined expected limit for the $\tau_{\mu}\tau_{h}$, $\tau_{h}\tau_{h}$, $\tau_{e}\tau_{h}$, $\tau_{e}\tau_{\mu}$ channels.}
%  \caption{Combined expected limit for the $\tau_{\mu}\tau_{h}$, $\tau_{h}\tau_{h}$, $\tau_{e}\tau_{h}$, $\tau_{e}\tau_{\mu}$ channels. A k factor of 1.3 has been used to scale the leading order (LO) signal cross-section.}
%  \label{fig:CombinedLimits}
%\end{figure}
\fi

