\section{Electron + Hadronic Tau Channel}\label{sec:eTauhad}
%\textbf{Responsible: E. Laird, Z. Mao}


\subsection{Event selection}\label{sec:et_selection}

Events must fire the single-electron trigger described in
Section~\ref{sec:triggers}.  We select reconstructed electrons
satisfying $p_T>35~\gev$, $\vert\eta\vert<2.1$, with a distance of
closest approach to the leading sum-$p_T^2$ primary vertex of less
than 0.045~cm (transvese) and 0.2~cm (longitudinal), passing the tight
working point of the e/$\gamma$ POG non-triggering MVA ID, having no
matched conversion nor missing hits, and within $\Delta R<0.5$ of the
HLT electron that fired the trigger.

Offline $\tauh$'s are required to have $p_{T}> 20~\gev$ and $\vert
\eta \vert < 2.1$, have a distance of closest approach to the leading
sum-$p_T^2$ primary vertex of less than 0.2~cm (longitudinal), pass
the new Decay Mode Finding requirement as either a 1-prong or 3-prong
$\tauh$, and pass the ``againstElectronVLooseMVA5'' and
``againstMuonTight3'' identification requirements.
%The isolation used is ``CombinedIsolationDeltaBetaCorr3Hits.''

We build pairs of electrons and $\tauh$'s in which the electron and
$\tauh$ are separated by at least $\Delta R > 0.5$.  In events with
more than one such pair, we select the pair with the two most isolated
leptons, considering first the electron, and then the $\tauh$.  This
criterion was seen to have good efficiency for signal samples.  In the
rare case of multiple such pairs having identical isolation values,
the reconstructed $p_T$'s are considred, preferring higher values.

%% Following the ID, pt/eta cuts, trigger matching and $dR$ condition
%% there may still be more than one possible candidate pair. We resolve
%% this by selecting the pair with the two most isolated leptons, under
%% the assumption that this gives the highest efficiency for selecting
%% the correct pair in signal events.  The following is the logic for
%% comparing two pairs in a sorting algorithm that aims to resolve
%% ambiguous cases (e.g. multiple candidates with the same isolation
%% value).\par First prefer the pair with the most isolated candidate 1
%% (muon for $e\mu$ and electron for $e\tau$).\par If the isolation of
%% candidate 1 is the same in both pairs, prefer the pair with the
%% highest candidate 1 pt (for cases of genuinely the same isolation
%% value but different possible candidate 1).\par If the pt of candidate
%% 1 in both pairs is the same (likely because it's the same object) then
%% prefer the pair with the most isolated candidate 2 (tau for $e\tau$
%% and electron for $e\mu$).\par If the isolation of candidate 2 is the
%% same, prefer the pair with highest candidate 2 pt (for cases of
%% genuinely the same isolation value but different possible candidate
%% 2).

After a pair has been chosen for an event, we apply the following
isolation requirements on the leptons, for an event to enter the
signal region: electron relative isolation $<0.15$; \tauh isolation
``byTightCombinedIsolationDeltaBetaCorr3Hits.''  In order to keep the
different final states exclusive, an event is rejected if there is an
additional electron satisfying the above identification requirements
and with relative isolation $<0.3$, or a muon satisfying the
identification requirements described in
Section~\ref{sec:em_selection} with relative isolation $<0.3$.  To
reduce further possible di-electron events in the $\teth$ channel, an
event is rejected if there is an opposite-charge electron pair with
$\Delta R > 0.15$ in which both of the electrons satisfy $p_T >
15~\gev$, $\vert \eta \vert < 2.5$, the e/$\gamma$ POG ``veto'' ID,
and relative isolation $<0.3$.

As for the other channels, the signal region is defined as having
\begin{itemize}
  \item $\cos{\Delta \phi (e,\tau_{h})}<-0.95$
  \item $Q(e) \times Q(\tau_{h}) < 0 $
  \item $\ETslash>30~\gev$
  \item $P_{\zeta}- 3.1 \times P_{\zeta}^{vis} > -50~\gev$
  \item no jet with $p_T>30\gev$ tagged as a b-jet (CSV loose)
\end{itemize}

The Standard Model processes considered as backgrounds are Drell-Yan,
di-boson production, top quark single and pair production, W+jets
production, and QCD multi-jet production.

\subsection{Genuine dilepton events}

Studies of simulated events indicate that for Drell-Yan process, top
quark single and pair production, and di-boson production, the
reconstructed and selected electrons and hadronic taus are typically
associated with genuine simulated leptons.  The nominal expected event
rates are estimated by scaling the simulated samples by the best
available cross sections, listed in Table~\ref{tab:mc_samples}, and by
the integrated luminosity of the data samples.

\subsubsection{Drell-Yan process}
Due to large W+Jets and QCD contamination, as shown in the left panel of 
Figure~\ref{fig:et_dy_tt} with the following selections, we use the 
Drell-Yan rates systematics (12\%) estimated from the \tetm final state in 
~\ref{sec:em_DY}.
\begin{itemize}
  \item $\ETslash<30~\gev$
  \item no jet with $p_T>30\gev$ tagged as a b-jet (CSV loose)
  \item \meffetau $<$ 200 GeV
\end{itemize}

\subsubsection{$t\bar{t}$ and single top processes process}
Systematics for $t\bar{t}$ and single top processes are estimated in an top rich 
region with the following selections and shown in the right panel of 
Figure~\ref{fig:et_dy_tt}:
\begin{itemize}
  \item $\cos{\Delta \phi (e,\tau_{h})}<-0.95$
  \item $\ETslash>30~\gev$
  \item $P_{\zeta}- 3.1 \times P_{\zeta}^{vis} > -50~\gev$
  \item at least one jet with $p_T>30\gev$ tagged as a b-jet (CSV loose)
\end{itemize}
The $t\bar{t}$ + single top production rate systematics estimated to be:
\begin{equation}\label{eq:tt}
\text{$t\bar{t}$ + single top systematics} = \left| 1 - \frac{\text{$t\bar{t}$ + single top}}{\text{Data - other backgrounds}}\right| = 5\%
\end{equation}
For \teth, we take the $t\bar{t}$ + single top production rate systematics (8\%)
estimated in the \tetm final state in ~\ref{sec:em_tt} which has higher top purity
and a more conservative value.

\begin{figure}\centering
  \includegraphics[width=0.45\textwidth]{figures/et-em/Data_MC_Comparison/et_DY_Validation_Met_lessThan30_nCSVL_lessThan1_closeUp}
  \includegraphics[width=0.45\textwidth]{figures/et-em/eTauStudies/ttbar_check}
  \caption{\label{fig:et_dy_tt} Distributions of \meffetau. Left:
    validation region with $\ETslash<30~\gev$, $n_b = 0$ \meffetau $<$ 200 GeV.  Right:
    validation region with $n_b\geq1$.}
\end{figure}

\subsubsection{Di-boson process}
We take di-boson processes directly from simulation with a 15\% production uncertainty.

    
\subsection{QCD multi-jet background}\label{sec:etau_qcd}
For a given variable and binning, e.g. the effective mass variable
used for signal extraction, we construct a data-driven template for
the shape of the QCD multi-jet background, i.e. the processes lacking
prompt leptons.  Based on the charge of the final state and \tauh isolation,
we split the events into four regions shown in Figure~\ref{fig:ABCD} and 
described below:
\begin{itemize}
  \item A (Signal) Region: e and \tauh have opposite charge and \tauh pass "Tight" isolation requirement.
  \item B Region: e and \tauh have same charge and \tauh pass "Tight" isolation requirement.
  \item C Region: e and \tauh have opposite charge and \tauh pass anti-isolation requirement.
  \item D Region: e and \tauh have same charge and \tauh pass anti-isolation requirement.
\end{itemize}

\begin{figure}\centering
  \includegraphics[width=0.45\textwidth]{figures/et-em/figs/ABCD}
  \caption{\label{fig:ABCD} QCD estimation and validation strategy for the \teth channel.}
\end{figure}
In each region (B, C, D), QCD events are estimated by subtracting events
with genuine leptons (estimated by simulation) bin-by-bin from data.
QCD events are assumed to be charge blind, thus, the amount of QCD events in region B should be 
comparable to that of in the signal region. However, with the freedom to define the
anti-isolation region, we choose an anti-isolation definition such
that there are much more QCD in region C compared to the signal region. 
Taking the QCD shape from region C will provide us 
a much smoother template for QCD estimation.

Hence, QCD events in the signal region are estimated with the shape from region C 
and multiplying a scale factor derived from regions B and D. The factor is defined as:
\begin{equation}\label{eq:et_qcd_sf}
f_\mathrm{LT}^\mathrm{QCD} = \left(N_\mathrm{data}^\mathrm{B} - N_\mathrm{MC}^\mathrm{B}\right)
/ \left(N_\mathrm{data}^\mathrm{D} -
N_\mathrm{MC}^\mathrm{D}\right)\quad.
\end{equation}

This QCD estimation method is valid only if the QCD shape in anti-isolated 
correctly models the QCD shape in the isolated region. The check is done 
by comparing the observation and background estimation in region B with 
the QCD shape taking from region D and normalized to the QCD in region B. 
An example of this test is shown in the right panel of Figure~\ref{fig:sst} 
with the anti-isolation definition as: \tauh isolation failing the "Tight" 
working point but below 5.0\gev. 

With the freedom to define the \tauh anti-isolation region, before the 
signal region selections so there is enough statistics, we scan 
through different anti-isolation definitions as shown in Figure~\ref{fig:et_scans} 
where the x-axis labels the start of the anti-isolation region and the 
y-axis labels the end. With each anti-isolation definition, we perform 
the check mentioned above and calculate $\chi^2$ between observation 
and background estimation. The p-value of the $\chi^2$ tests for each 
anti-isolation definition is shown on the cells in the right panel 
of Figure~\ref{fig:et_scans}. The left panel of Figure~\ref{fig:et_scans} shows 
the same test but with $f_\mathrm{LT}^\mathrm{QCD}$ instead in each cell.

Based on the following criteria:
\begin{itemize}
  \item good results from $\chi^2$ test
  \item low "Loose-to-Tight" scale factor ($f_\mathrm{LT}^\mathrm{QCD}$) to ensure a smooth template
\end{itemize}

The range of isolation failing the tight working point, but below 5.0\gev, was chosen as the
sideband with. After the signal region selection the "Loose-to-Tight" scale factor is estimated
to be: $0.13 \pm 0.02$ where an additional 15\% uncertainty is added to the QCD systematics 
on top of the bin-by-bin systematics.

\begin{figure}\centering
  \includegraphics[width=0.45\textwidth,page=68]{figures/et-em/antiIsolationScan/et_puweight_chi2_SS_123prong_antiIsoScan} %% et_SS_chi2Scan_SF
  \includegraphics[width=0.45\textwidth,page=67]{figures/et-em/antiIsolationScan/et_puweight_chi2_SS_123prong_antiIsoScan} %% et_SS_chi2Scan_p_value
  \caption{\label{fig:et_scans} \teth channel: scan of the range of
    relaxed isolation for the $\tau_h$.  Left: text within each bin
    gives the normalization factor applied to same-charge iso-relaxed
    events (``loose to tight'' factor); the color axis matches the
    right plot.  Right: $\chi^2$-probability of the agreement between
    the predicted and observed distributions of tightly-isolated
    same-charge events.}
\end{figure}

\begin{figure}\centering
  \includegraphics[width=0.45\textwidth]{figures/et-em/antiIsolationScan/em_SST}
  \includegraphics[width=0.45\textwidth,page=6]{figures/et-em/antiIsolationScan/et_puweight_chi2_SS_123prong_antiIsoScan} %% et_SST
  \caption{\label{fig:sst} The distributions of reconstructed parent
    mass in the same-charge, tightly-isolated sample.  Left: \tetm
    channel: \meffemu.  Right: \teth channel: \meffetau.}
\end{figure}


\subsection{W+jets and validation of total estimated background}
\label{sec:et_w_bkg_validation}

The simulated W+jets samples, especially at low \HT, were not
generated with large statistics.  If used directly, avoiding
non-smooth templates restricts somewhat the choice of signal selection
and binning.  It also complicates the validation of the background
estimates.  As a workaround, we use a region of relaxed \tauh
isolation to obtain the simulated shape, and normalize it to the
integrated MC yield when requiring tight \tauh isolation.
Figure~\ref{fig:et-w-shape} compares the direct and relaxed
predictions in the signal region and in the control region discussed
below. 

\begin{figure}\centering
  \includegraphics[width=0.45\textwidth]{figures/et-em/WJetsClosureTest/highMETlowPZeta0BTag_m_withMET}
  %% \includegraphics[width=0.45\textwidth]{figures/et-em/WJetsClosureTest/highMETlowPZeta0BTag_mt}
  \includegraphics[width=0.45\textwidth]{figures/et-em/WJetsClosureTest/signalRegionSelection_m_withMET}
  \caption{\label{fig:et-w-shape} Left: comparison of the simulated
    distributions of \meffetau in the signal region, with tight \tauh
    isolation and with relaxed \tauh isolation.  Right: analogous
    comparison for the control region described in Section~
    \ref{sec:et_w_bkg_validation}.}
\end{figure}


To further check the dependency of reconstructed mass shape on
\tauh isolation, Figure~\ref{fig:et-w-shape2} compares the reconstructed
mass shape in \tauh isolation sideband 1 with \tauh isolation sideband 2 (defined below).

\tauh isolation sideband 1: \tauh failing "Tight" working point while \tauh iso $<$ 2.5;

\tauh isolation sideband 2: 2.5 $<$ \tauh iso $<$ 5;

\begin{figure}\centering
  \includegraphics[width=0.45\textwidth]{figures/et-em/eTauStudies/WJets_Tight_to_2p5_vs_2p5_5}
  \caption{\label{fig:et-w-shape2} comparison of the simulated
    distributions of \meffetau in the signal region, with \tauh in isolation sideband 1 and 
\tauh in isolation sideband 2.}
\end{figure}

To evaluate the estimated background rate in a signal-depleted and
W-enriched region, we make the same requirements as for the signal
region, except require $p_{\zeta} - 3.1 p_{\zeta\mathrm{vis.}}<
-50~\gev$ (i.e., inverted w.r.t. the signal selection), and allowing
any value of $\cos{\Delta\phi_{e\tauh}}$ (i.e., relaxed w.r.t. the
signal selection).  We then scan the W+jets event rate from 1.0 to
1.25, in steps of 0.05.  For each assumed rate, we redetermine the QCD
prediction in this control region.  Figure~\ref{fig:etau_w_sf} shows
the agreement between observations and background estimates at each
scan point.  A scale factor of $1.15\pm0.10$ gives the best agreement
in this control region.  The distribution of \meffetau in the
W-enriched control region, requiring also
$\cos{\Delta\phi_{e\tauh}}<-0.8$ (to be somewhat closer to the signal
region), is shown in Figure~\ref{fig:et_meff_flipped_pz}.

\begin{figure}\centering
  \includegraphics[width=0.3\textwidth,page=5]{figures/et-em/WJetsSFScan/et_puweight_OS_signalRegionReversePZeta_1and3prong_mt}
  \includegraphics[width=0.3\textwidth,page=6]{figures/et-em/WJetsSFScan/et_puweight_OS_signalRegionReversePZeta_1and3prong_mt}
  \includegraphics[width=0.3\textwidth,page=7]{figures/et-em/WJetsSFScan/et_puweight_OS_signalRegionReversePZeta_1and3prong_mt}\\
  \includegraphics[width=0.3\textwidth,page=8]{figures/et-em/WJetsSFScan/et_puweight_OS_signalRegionReversePZeta_1and3prong_mt}
  \includegraphics[width=0.3\textwidth,page=9]{figures/et-em/WJetsSFScan/et_puweight_OS_signalRegionReversePZeta_1and3prong_mt}
  \includegraphics[width=0.3\textwidth,page=10]{figures/et-em/WJetsSFScan/et_puweight_OS_signalRegionReversePZeta_1and3prong_mt}
  \caption{\label{fig:etau_w_sf} Distributions of
    $m_\mathrm{T}\left(e,\ETslash\right)$ obtained in the W-enriched
    control region, when scanning the W+jets event rate from 1.0 to
    1.25 times its nominal value, in steps of 0.05.  For each assumed
    W+jets event rate, the QCD prediction is determined anew using the
    procedure described in Section~\ref{sec:etau_qcd}.}
\end{figure}

\begin{figure}\centering
  \includegraphics[width=0.45\textwidth]{figures/et-em/Data_MC_Comparison/et_puweight_OS_signalRegionReversePZeta_1and3prong}
  \caption{\label{fig:et_meff_flipped_pz} The distributions of
    reconstructed parent mass in the validation region, \teth channel:
    \meffetau.}
\end{figure}

The expected SM event yields in the signal region, in a fixed-width
binning, are shown in Figure~\ref{fig:etau_sm_template}.
\begin{figure}\centering
  \includegraphics[width=0.45\textwidth]{figures/et-em/backgroundTemplates/et_puweight_OS_signalRegion_1and3prong}
  \caption{\label{fig:etau_sm_template} Expected event yields for the
    SM processes in the \teth channel.  The data histogram is not
    shown in order to remain ``blind.''}
\end{figure}



\subsection{Background checks in data excess region}
\label{sec:bkg_validation_in_data_excess_region}

\begin{figure}\centering
  \includegraphics[width=0.45\textwidth]{figures/et-em/eTauStudies/excessCloseUp}
  \caption{\label{fig:closeUp} The distributions of
    reconstructed parent mass between 300 to 600 GeV, \teth channel:
    \meffetau.}
\end{figure}

The background composition for \meffetau in region 
[300, 600] is shown in Figure~\ref{fig:closeUp}. As shown, the dominating 
background in this region is W+Jets, consisting of around 55\% of the total 
background, followed by QCD accounting for 19\% and $t\bar{t}$ accounting for 16\%
of the total background.

\subsubsection{W+Jets checks}
\begin{figure}\centering
  \includegraphics[width=0.45\textwidth]{figures/et-em/eTauStudies/mt_distribution}
  \caption{\label{fig:mT} The distributions of
    transverse mass, \teth channel:$m_T$}
\end{figure}

Addition to the checks mentioned in ~\ref{sec:et_w_bkg_validation}, based on $m_T$ 
distribution as shown in Figure~\ref{fig:mT}, we require $60~\text{GeV} < m_T < 120~\text{GeV}$
to compare data v.s estimated background in a W+Jets rich region. From 
Figure~\ref{fig:60_mT_120} it is easy to conclude that the W+Jets shape 
estimated from the relaxed \tauh isolation region provides a much
smoother template and agrees much better with observations.

\begin{figure}\centering
  \includegraphics[width=0.45\textwidth]{figures/et-em/eTauStudies/WJets_non_Iso_60_mt_120}
  \includegraphics[width=0.45\textwidth]{figures/et-em/eTauStudies/WJets_iso_60_mt_120}
  \caption{\label{fig:60_mT_120} Left: The distributions of
    reconstructed parent mass with W+Jets shape estimated from
    relaxed \tauh isolation region. Right: The distributions of
    reconstructed parent mass with W+Jets shape estimated from
    tight \tauh isolation region., \teth channel:
    \meffetau.}
\end{figure}

A direct comparison of the W+Jets shape is shown in Figure~\ref{fig:WJets_60_mT_120}
where we can see the significant upwards fluctuation of W+Jets in the tight \tauh 
isolation region for \meffetau regions of [300, 600].

\begin{figure}\centering
  \includegraphics[width=0.45\textwidth]{figures/et-em/eTauStudies/WJets_iso_vs_non-Iso_60_mt_120}
  \caption{\label{fig:WJets_60_mT_120} Comparison of the simulated
    distributions of \meffetau in the signal region requiring 
    $60~\text{GeV} < m_T < 120~\text{GeV}$, with tight \tauh
    isolation and with relaxed \tauh isolation.
    \teth channel:\meffetau}
\end{figure}

For $m_T > 120~\text{GeV}$, W+Jets shape comparisons and distributions of \meffetau are shown
in Figure~\ref{fig:120_mT}. From the plot on the left, we can see the W+Jets shape from relaxed 
and tight \tauh isolation region agrees well in high mass regions. Due to low statistics, it 
is difficult to conclude if there are any excess in \meffetau regions [300, 600]. Overall, 
W+Jets shape estimated from the relaxed \tauh isolation region agrees better with observations.

\begin{figure}\centering
  \includegraphics[width=0.32\textwidth]{figures/et-em/eTauStudies/WJets_iso_vs_non-Iso_120_mt}
  \includegraphics[width=0.32\textwidth]{figures/et-em/eTauStudies/WJets_non_Iso_120_mt}
  \includegraphics[width=0.32\textwidth]{figures/et-em/eTauStudies/WJets_iso_120_mt}
  \caption{\label{fig:120_mT} In the signal region requiring 
    $m_T > 120~\text{GeV}$: Left: Comparison of the simulated
    distributions of \meffetau, with tight \tauh isolation and 
    with relaxed \tauh isolation. Center: The distributions of
    reconstructed parent mass with W+Jets shape estimated from
    relaxed \tauh isolation region. Right: The distributions of
    reconstructed parent mass with W+Jets shape estimated from
    tight \tauh isolation region. \teth channel}
\end{figure}

In the signal rich region, with $m_T < 60~\text{GeV}$, the left plot
in Figure~\ref{fig:mT_60} shows the good agreement between the W+Jets 
shape from relaxed and tight \tauh isolation regions. The center and right
plot further shows that the excess seen in the high \meffetau region cannot be
explained by W+Jets MC shape variations between relaxed and tight \tauh
isolation regions

\begin{figure}\centering
  \includegraphics[width=0.32\textwidth]{figures/et-em/eTauStudies/WJets_iso_vs_non-Iso_mt_60}
  \includegraphics[width=0.32\textwidth]{figures/et-em/eTauStudies/WJets_non_Iso_mt_60}
  \includegraphics[width=0.32\textwidth]{figures/et-em/eTauStudies/WJets_iso_mt_60}
  \caption{\label{fig:mT_60} In the signal region requiring 
    $m_T < 60~\text{GeV}$: Left: Comparison of the simulated
    distributions of \meffetau, with tight \tauh isolation and 
    with relaxed \tauh isolation. Center: The distributions of
    reconstructed parent mass with W+Jets shape estimated from
    relaxed \tauh isolation region. Right: The distributions of
    reconstructed parent mass with W+Jets shape estimated from
    tight \tauh isolation region. \teth channel}
\end{figure}

2-prong \tauh's were rejected in the \teth due low signal presence and
high W+Jets and QCD contamination. Hence, this region is a perfect testing area
for our background estimation methods. As shown in Figure~\ref{fig:2prong}
there is a good agreement between data and estimated background. Which further
validates our modeling of the W+Jets and QCD background. 

\begin{figure}\centering
  \includegraphics[width=0.45\textwidth]{figures/et-em/eTauStudies/2prong}
  \caption{\label{fig:2prong} In the signal region requesting
    2 prong \tauh, the distributions of reconstructed parent mass with
    W+Jets shape estimated from relaxed \tauh isolation region.
    \teth channel: \meffetau}
\end{figure}

\subsubsection{QCD checks}
From prior checks we have concluded that the W+Jets \meffetau shape is mostly
independent of the \tauh isolation region, it is obvious to ask the same
question for QCD. Figure~\ref{fig:QCD_check_1} compares the data and estimated
background with W+Jets and QCD shape take from \tauh isolation region 1' and 
\tauh isolation region 2' as explained below:

\tauh isolation region 1': \tauh failing "Tight" working point while \tauh iso $<$ 5;

\tauh isolation region 2': \tauh failing "Medium" working point while \tauh iso $<$ 4;

\begin{figure}\centering
  \includegraphics[width=0.45\textwidth]{figures/et-em/eTauStudies/QCD_from_tight_5}
  \includegraphics[width=0.45\textwidth]{figures/et-em/eTauStudies/QCD_from_medium_4}
  \caption{\label{fig:QCD_check_1} In the signal region, Left: distributions of 
    reconstructed parent mass with W+Jets and QCD shape estimated from 
    \tauh isolation region 1'. Right: distributions of 
    reconstructed parent mass with W+Jets and QCD shape estimated from 
    \tauh isolation region 2'. 
    \teth channel: \meffetau}
\end{figure}

Here, we see the change in \tauh isolation region definition does not effect 
the overall background estimation.


Since W+Jets events in the \teth channel most likely consists of one real electron
while QCD events consists of only jets faking electrons and \tauh's. We can create
separate W+Jets and QCD rich regions by applying different electron isolation cuts.
Figure~\ref{fig:eRelIso} shows the data and estimated background comparison in
W+Jets rich regions (electron relIso $< 0.01$ and $0.01<$ electron relIso $< 0.04$)
and QCD rich region ($0.04<$ electron relIso $< 0.15$). Here, we see a good
agreement between data and estimated background in the QCD rich region. An enhanced
discrepancy between data and estimated background can be seen for regions with
high signal expectancy.  

\begin{figure}\centering
  \includegraphics[width=0.32\textwidth]{figures/et-em/eTauStudies/e_0p01}
  \includegraphics[width=0.32\textwidth]{figures/et-em/eTauStudies/e_0p01_0p04}
  \includegraphics[width=0.32\textwidth]{figures/et-em/eTauStudies/e_0p04_0p15}
  \caption{\label{fig:eRelIso} In the signal region, Left: distributions of 
    reconstructed parent mass requiring eRelIso < 0.01. Center: distributions of 
    reconstructed parent mass requiring 0.01 < eRelIso < 0.04. Right: 
    distributions of reconstructed parent mass requiring 0.04 < eRelIso < 0.15. 
    \teth channel: \meffetau}
\end{figure}


\subsubsection{$t\bar{t}$ checks}
A $t\bar{t}$ rich region was constructed by reverting the b-veto requirement with
the selections below:
\begin{itemize}
  \item $\cos{\Delta \phi (e,\tau_{h})}<-0.95$
  \item $Q(e) \times Q(\tau_{h}) < 0 $
  \item $\ETslash>30~\gev$
  \item $P_{\zeta}- 3.1 \times P_{\zeta}^{vis} > -50~\gev$
  \item at least one jet with $p_T>30\gev$ tagged as a b-jet (CSV loose)
\end{itemize}

Figure~\ref{fig:ttbar} shows a good agreement of data and estimated background in this region.

\begin{figure}\centering
  \includegraphics[width=0.45\textwidth]{figures/et-em/eTauStudies/ttbar_check}
  \caption{\label{fig:ttbar} In the signal region with at least 
    one jet with $p_T>30\gev$ tagged as a b-jet (CSV loose),
    the distributions of reconstructed parent mass with
    W+Jets shape estimated from relaxed \tauh isolation region.
    \teth channel: \meffetau}
\end{figure}

