\subsection{Jet Reconstruction}
%\textbf{Responsible: Z.Mao E.Laird}

In this analysis, jets, especially b-tagged jets, are used for two purposes: to reduce $\rm{t\bar{t}}$ background in the signal region and to obtain $\rm{t\bar{t}}$ enriched 
control samples used to check the $\rm{t\bar{t}}$ modeling.

The particle-flow (PF) technique~\cite{CMS-PAS-PFT-09-001,CMS-PAS-PFT-10-002} is used to improve the jet $p_T$ and angular resolution in this analysis.
The PF technique combines information from different subdetectors to produce a mutually exclusive collection of particles (namely muons, electrons, photons, 
charged hadrons and neutral hadrons) that are used as input for the jet clustering algorithms.
Jets are clustered using the anti-$k_{T}$ algorithm~\cite{anti-kT}, with a distance parameter of $R$ = 0.4 in $\eta$-$\phi$ plane (defined as $\Delta R = 
\sqrt{\Delta \eta^2 + \Delta \phi^2}$) by summing the four-momenta of individual PF particles.

The jets require energy corrections obtained using simulated events that are generated with \texttt{PYTHIA}, processed through a detector simulation based on 
\texttt{GEANT4}, and confirmed with in situ measurements of the $p_T$ balance.
The overall jet-energy corrections depend on the $\eta$ and $p_T$ values of jets.
The jet-energy corrections are applied to the following levels: L1 corrections; L2L3 MC-truth corrections and L2L3 residuals corrections (applied to data only). 
The L1 corrections use the event-by-event UE/PU densities to remove the energy coming from pile-up events.
The L2L3 MC-truth corrections improve the energy response as a function of jet $p_T$ and $\eta$.
The L2L3 residuals corrections, applied only to data, corrects the remaining difference within 
the jet response between data and MC.

In this analysis, the latest jet energy corrections (JEC) provided by the POG are applied.

Jets are required to have $p_T >$ 30 GeV and $|\eta| <$ 2.4.
For the identification of jets the loose ID working point is used in this analysis.
Table 4 shows the selection criteria used for the recommended loose ID, which are validated in other studies~\cite{CMS-PAS-FSQ-12-035}.
The jet reconstruction and ID efficiency in simulation is $>$98\%. Table 4 does not include the most recent requirements 
for $2.7 < \eta < 3$ since the jet ID in this analysis is used only in the context of those jets which will also be considered for b-tagging 
(i.e. b-tagging only considers jets with $\eta < 2.4$).

\begin{table}[ht]
\begin{center}
 \caption{Loose Jet-ID Selections.\label{tab:jetId}}
 \begin{tabular}{|cc|}
 \hline\hline
       Selection                        & Cut        \\[0.5ex] \hline
       Neutral Hadron Fraction          & $<0.99$      \\
       Neutral EM Fraction              & $<0.99$      \\
       Number of Constituents           & $> 1$        \\
       And for $\eta < 2.4$ , $\eta > -2.4$ in addition apply &\\
       Charged Hadron Fraction 	        & $> 0$   \\
       Charged Multiplicity             & $> 0$   \\
       Charged EM Fraction              & $<0.99$ \\
 \hline
 \hline
 \end{tabular}
\end{center}
\end{table}


\subsubsection{b-Jet Tagging}

The CSVv2 algorithm~\cite{Chatrchyan:2012jua} is used to identify jet as originating from hadronization of a b-quark. This algorithm combines 
reconstructed secondary vertex and track-based lifetime information to build a likelihood-based discriminator to distinguish between jets from b-quarks and those 
from charm or light quarks and gluons.

The minimum thresholds on these discriminators define loose, medium, and tight operating points with a misidentification probability of about 10\%, 1\%, and 
0.1\%, respectively, with an average jet $p_T$ of about 80 GeV. The loose operating point, which provides an efficiency of around 80\%, is used in this analysis.
A large sample of pair-produced top quark events is used to measure b-tagging efficiency using several methods~\cite{CMS-PAS-BTV-13-001}. 
Following the POG recommendation, a scale factor is applied to correct for differences in b-tagging efficiency between data and simulation~\cite{bTagging}.
